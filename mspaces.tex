\documentclass[12pt]{book}
%\usepackage{pdf14}
%Paper saving
%\documentclass[12pt,openany]{book}
%\documentclass[10pt,openany]{book}
%\documentclass[8pt,openany]{extbook}

\usepackage[T1]{fontenc}

\usepackage{enumerate}
\usepackage{ifpdf}
\usepackage{amsmath}
\usepackage[psamsfonts]{amsfonts}
\usepackage[psamsfonts]{amssymb}
\usepackage{amsthm}
\usepackage[pdftex]{graphicx}
%\usepackage{color}
%\usepackage{graphics}
\usepackage[headings]{fullpage}
\usepackage{url}
\usepackage{varioref}
\usepackage{floatflt}
\usepackage{wrapfig}
\usepackage{makeidx}
\usepackage[pdftex]{hyperref}
\usepackage[all]{hypcap}
\usepackage[shortalphabetic]{amsrefs}
\usepackage[all]{xy}
\usepackage{nicefrac}
\usepackage{microtype}

\usepackage{tikz}
\usepackage{rotating}



% Times
%\usepackage{txfonts}
% Times, but symbol/cm/ams math fonts
\usepackage{mathptmx}
% But we do want helvetica for sans
\usepackage{helvet}


% useful
\newcommand{\ignore}[1]{}

% analysis/geometry stuff
\newcommand{\ann}{\operatorname{ann}}
\renewcommand{\Re}{\operatorname{Re}}
\renewcommand{\Im}{\operatorname{Im}}
\newcommand{\Orb}{\operatorname{Orb}}
\newcommand{\hol}{\operatorname{hol}}
\newcommand{\aut}{\operatorname{aut}}
\newcommand{\codim}{\operatorname{codim}}
\newcommand{\sing}{\operatorname{sing}}

% reals
\newcommand{\esssup}{\operatorname{ess~sup}}
\newcommand{\essran}{\operatorname{essran}}
\newcommand{\innprod}[2]{\langle #1 | #2 \rangle}
\newcommand{\linnprod}[2]{\langle #1 , #2 \rangle}
\newcommand{\supp}{\operatorname{supp}}
\newcommand{\Nul}{\operatorname{Nul}}
\newcommand{\Ran}{\operatorname{Ran}}
\newcommand{\sabs}[1]{\lvert {#1} \rvert}
\newcommand{\snorm}[1]{\lVert {#1} \rVert}
\newcommand{\abs}[1]{\left\lvert {#1} \right\rvert}
\newcommand{\norm}[1]{\left\lVert {#1} \right\rVert}
\newcommand{\sgn}{\operatorname{sgn}}

% sets (some)
\newcommand{\C}{{\mathbb{C}}}
\newcommand{\R}{{\mathbb{R}}}
\newcommand{\Z}{{\mathbb{Z}}}
\newcommand{\N}{{\mathbb{N}}}
\newcommand{\Q}{{\mathbb{Q}}}
\newcommand{\D}{{\mathbb{D}}}
\newcommand{\F}{{\mathbb{F}}}

% consistent
\newcommand{\bB}{{\mathbb{B}}}
\newcommand{\bC}{{\mathbb{C}}}
\newcommand{\bR}{{\mathbb{R}}}
\newcommand{\bZ}{{\mathbb{Z}}}
\newcommand{\bN}{{\mathbb{N}}}
\newcommand{\bQ}{{\mathbb{Q}}}
\newcommand{\bD}{{\mathbb{D}}}
\newcommand{\bF}{{\mathbb{F}}}
\newcommand{\bH}{{\mathbb{H}}}
\newcommand{\bO}{{\mathbb{O}}}
\newcommand{\bP}{{\mathbb{P}}}
\newcommand{\bK}{{\mathbb{K}}}
\newcommand{\bV}{{\mathbb{V}}}
\newcommand{\CP}{{\mathbb{CP}}}
\newcommand{\RP}{{\mathbb{RP}}}
\newcommand{\HP}{{\mathbb{HP}}}
\newcommand{\OP}{{\mathbb{OP}}}
\newcommand{\sA}{{\mathcal{A}}}
\newcommand{\sB}{{\mathcal{B}}}
\newcommand{\sC}{{\mathcal{C}}}
\newcommand{\sF}{{\mathcal{F}}}
\newcommand{\sG}{{\mathcal{G}}}
\newcommand{\sH}{{\mathcal{H}}}
\newcommand{\sM}{{\mathcal{M}}}
\newcommand{\sO}{{\mathcal{O}}}
\newcommand{\sP}{{\mathcal{P}}}
\newcommand{\sQ}{{\mathcal{Q}}}
\newcommand{\sR}{{\mathcal{R}}}
\newcommand{\sS}{{\mathcal{S}}}
\newcommand{\sI}{{\mathcal{I}}}
\newcommand{\sL}{{\mathcal{L}}}
\newcommand{\sK}{{\mathcal{K}}}
\newcommand{\sU}{{\mathcal{U}}}
\newcommand{\sV}{{\mathcal{V}}}
\newcommand{\sX}{{\mathcal{X}}}
\newcommand{\sY}{{\mathcal{Y}}}
\newcommand{\sZ}{{\mathcal{Z}}}
\newcommand{\fS}{{\mathfrak{S}}}

\newcommand{\interior}{\operatorname{int}}

% Topo stuff
\newcommand{\id}{\textit{id}}
\newcommand{\im}{\operatorname{im}}
\newcommand{\rank}{\operatorname{rank}}
\newcommand{\Tor}{\operatorname{Tor}}
\newcommand{\Torsion}{\operatorname{Torsion}}
\newcommand{\Ext}{\operatorname{Ext}}
\newcommand{\Hom}{\operatorname{Hom}}

%extra thingies
\newcommand{\mapsfrom}{\ensuremath{\text{\reflectbox{$\mapsto$}}}}
\newcommand{\from}{\ensuremath{\leftarrow}}
\newcommand{\dhat}[1]{\hat{\hat{#1}}}

% San Serif fonts
%\renewcommand{\familydefault}{\sfdefault}

% To allow skrinking to 5.5 x 8.5 inches without whitespaces
% Make sure to rerun makeindex as well
% Useful for printing on lilu.com and saving on paper
%\addtolength{\textheight}{2.13in}
%\addtolength{\paperheight}{2.13in}

\hypersetup{
    %colorlinks,
    pdfborderstyle={/S/U/W 1},
    %citecolor=black,
    %filecolor=black,
    %linkcolor=black,
    %urlcolor=black,
    pdfkeywords={real analysis, Riemann integral, derivative, limit, sequence},
    pdfsubject={Real Analysis},
    pdftitle={Basic Analysis: Introduction to Real Analysis},
    pdfauthor={Jiri Lebl}
}

% Set up our index
\makeindex

% Very simple indexing
\newcommand{\myindex}[1]{#1\index{#1}}

% define this to be empty to kill notes
\newcommand{\sectionnotes}[1]{\noindent \emph{Note: #1} \medskip \par}

% Define this to be empty to not skip page before the sections to
% save some paper
\newcommand{\sectionnewpage}{\clearpage}
%\newcommand{\sectionnewpage}{}

\author{Ji\v{r}\'i Lebl}

\title{Basic Analysis: Introduction to Real Analysis}

% Don't include subsections
\setcounter{tocdepth}{1}

\theoremstyle{plain}
\newtheorem{thm}{Theorem}[section]
\newtheorem{lemma}[thm]{Lemma}
\newtheorem{prop}[thm]{Proposition}
\newtheorem{cor}[thm]{Corollary}

\theoremstyle{remark}
\newtheorem{remark}[thm]{Remark}

\theoremstyle{definition}
\newtheorem{defn}[thm]{Definition}

\newtheoremstyle{exercise}% name
  {}% Space above
  {}% Space below
  {\itshape}% Body font
  {}% Indent amount 1
  {\bfseries \itshape}% Theorem head font
  {:}% Punctuation after theorem head
  {.5em}% Space after theorem head 2
  {}% Theorem head spec (can be left empty, meaning "normal")


\theoremstyle{exercise}
\newtheorem{exercise}{Exercise}[section]

\newtheoremstyle{example}% name
  {}% Space above
  {}% Space below
  {}% Body font
  {}% Indent amount 1
  {\bfseries}% Theorem head font
  {:}% Punctuation after theorem head
  {.5em}% Space after theorem head 2
  {}% Theorem head spec (can be left empty, meaning "normal")

\theoremstyle{example}
\newtheorem{example}[thm]{Example}

% Footnotes should use symbols, not numbers.  Numbered footnotes are
% evil
\renewcommand{\thefootnote}{\fnsymbol{footnote}}

% referencing
\newcommand{\figureref}[1]{\hyperref[#1]{Figure~\ref*{#1}}}
\newcommand{\tableref}[1]{\hyperref[#1]{Table~\ref*{#1}}}
\newcommand{\chapterref}[1]{\hyperref[#1]{chapter~\ref*{#1}}}
\newcommand{\Chapterref}[1]{\hyperref[#1]{Chapter~\ref*{#1}}}
\newcommand{\sectionref}[1]{\hyperref[#1]{\S\ref*{#1}}}
\newcommand{\exerciseref}[1]{\hyperref[#1]{Exercise~\ref*{#1}}}
\newcommand{\exampleref}[1]{\hyperref[#1]{Example~\ref*{#1}}}
\newcommand{\thmref}[1]{\hyperref[#1]{Theorem~\ref*{#1}}}
\newcommand{\propref}[1]{\hyperref[#1]{Proposition~\ref*{#1}}}
\newcommand{\lemmaref}[1]{\hyperref[#1]{Lemma~\ref*{#1}}}
\newcommand{\corref}[1]{\hyperref[#1]{Corollary~\ref*{#1}}}
\newcommand{\defnref}[1]{\hyperref[#1]{Definition~\ref*{#1}}}

\begin{document}

\setcounter{chapter}{0}
\refstepcounter{chapter}
\label{rn:chapter}
\refstepcounter{chapter}
\label{seq:chapter}
\refstepcounter{chapter}
\label{lim:chapter}

\setcounter{chapter}{5}
\refstepcounter{chapter}
\label{fs:chapter}
%\setcounter{chapter}{6}

\let\oldchapter\chapter
\renewcommand*{\chapter}[1]{\oldchapter[#1]{#1 \hspace{\fill} {\rm \normalsize \today}}}

\newenvironment{exnote}{\small}{}

%%%%%%%%%%%%%%%%%%%%%%%%%%%%%%%%%%%%%%%%%%%%%%%%%%%%%%%%%%%%%%%%%%%%%%%%%%%%%%
%%%%%%%%%%%%%%%%%%%%%%%%%%%%%%%%%%%%%%%%%%%%%%%%%%%%%%%%%%%%%%%%%%%%%%%%%%%%%%
%%%%%%%%%%%%%%%%%%%%%%%%%%%%%%%%%%%%%%%%%%%%%%%%%%%%%%%%%%%%%%%%%%%%%%%%%%%%%%

\chapter{Metric Spaces} \label{ms:chapter}

%%%%%%%%%%%%%%%%%%%%%%%%%%%%%%%%%%%%%%%%%%%%%%%%%%%%%%%%%%%%%%%%%%%%%%%%%%%%%%

\section{Metric spaces}
\label{sec:metric}

\sectionnotes{1.5 lectures}

As mentioned in the introduction, the main idea in analysis is to take
limits.  In \chapterref{seq:chapter} we learned to take limits of sequences of
real numbers.  And in \chapterref{lim:chapter} we learned to take limits
of functions as a real number approached some other real number.

We want to take limits in more complicated contexts.  For
example, we might want to have sequences of points in 3-dimensional space.
Or perhaps we wish to define continuous functions of several variables.
We might even want to define functions on spaces that are a little harder to
describe, such as the surface of the earth.  We still want to talk about
limits there.

Finally, we have seen the limit of a sequence of
functions in \chapterref{fs:chapter}.
We wish to unify all these notions so that we do not have to
reprove theorems over and over again in each context.  The concept of a
metric space is an elementary yet powerful tool in analysis.  And while it
is not sufficient to describe every type of limit we can find in modern
analysis, it gets us very far indeed.

\begin{defn}
Let $X$ be a set and let
$d \colon X \times X \to \R$
be a function such that
\begin{enumerate}[(i)]
\item \label{metric:pos} $d(x,y) \geq 0$ for all $x, y$ in $X$,
\item \label{metric:zero} $d(x,y) = 0$ if and only if $x = y$,
\item \label{metric:com} $d(x,y) = d(y,x)$, 
\item \label{metric:triang} $d(x,z) = d(x,y)+ d(y,z)$ \qquad (\emph{\myindex{triangle inequality}}).
\end{enumerate}
Then the pair $(X,d)$ is called a \emph{\myindex{metric space}}.  The
function $d$ is called the \emph{\myindex{metric}} or sometimes the
\emph{\myindex{distance function}}.
Sometimes we just say $X$ is a metric space if the metric is clear from
context.
\end{defn}

The geometric idea is that $d$ is the distance between two points. 
Items \eqref{metric:pos}--\eqref{metric:com} have obvious geometric
interpretation: distance is always nonnegative, the only point that is
distance 0 away from $x$ is $x$ itself, and finally that the distance from
$x$ to $y$ is the same as the distance from $y$ to $x$.  The triangle
inequality \eqref{metric:triang} has the interpretation given in
\figureref{fig:mstriang}.
\begin{figure}[h!t]
\begin{center}
\input ms-triang.pdf_t
\caption{Diagram of the triangle inequality in metric spaces.\label{fig:mstriang}}
\end{center}
\end{figure}

For the purposes of drawing, it is convenient to draw figures and
diagrams in the plane and have the metric be the standard distance.
However, that is only one particular metric space.  Just because a
certain fact seems to be clear from drawing a picture does not mean it is
true.  You might be getting sidetracked by intuition from euclidean
geometry,
whereas the concept of a metric space is a lot more general.

Let us give some examples of metric spaces.

\begin{example}
The set of real numbers $\R$ is a metric space with the metric
\begin{equation*}
d(x,y) := \abs{x-y} .
\end{equation*}
Items \eqref{metric:pos}--\eqref{metric:com} of the definition
are easy to verify.  The
triangle inequality \eqref{metric:triang} follows immediately
from the standard triangle inequality for real numbers:
\begin{equation*}
d(x,z) = \abs{x-z} = 
\abs{x-y+y-z} \leq
\abs{x-y}+\abs{y-z} =
d(x,y)+ d(y,z) .
\end{equation*}
This metric is the \emph{\myindex{standard metric on $\R$}}.  If we talk
about $\R$ as a metric space without mentioning a specific metric, we 
mean this particular metric.
\end{example}

\begin{example}
We can also put a different metric on the set of real numbers.
For example take the set of real numbers $\R$ together with the metric
\begin{equation*}
d(x,y) :=
\frac{\abs{x-y}}{\abs{x-y}+1} .
\end{equation*}
Items \eqref{metric:pos}--\eqref{metric:com} are again easy to verify.  The
triangle inequality \eqref{metric:triang} is a little bit more difficult.
Note that $d(x,y) = \varphi(\abs{x-y})$ where $\varphi(t) =
\frac{t}{t+1}$ and note that $\varphi$ is an increasing function
(positive derivative) hence
\begin{equation*}
\begin{split}
d(x,z) & = \varphi(\abs{x-z}) = 
\varphi(\abs{x-y+y-z}) \leq
\varphi(\abs{x-y}+\abs{y-z})
\\
& =
\frac{\abs{x-y}+\abs{y-z}}{\abs{x-y}+\abs{y-z}+1} =
\frac{\abs{x-y}}{\abs{x-y}+\abs{y-z}+1} +
\frac{\abs{y-z}}{\abs{x-y}+\abs{y-z}+1}
\\
& \leq
\frac{\abs{x-y}}{\abs{x-y}+1} +
\frac{\abs{y-z}}{\abs{y-z}+1} =
d(x,y)+ d(y,z) .
\end{split}
\end{equation*}
Here we have an example of a nonstandard metric on $\R$.  With this metric
we can see for example that $d(x,y) < 1$ for all $x,y \in \R$.  That is,
any two points are less than 1 unit apart.
\end{example}

An important metric space is the
$n$-dimensional \emph{\myindex{euclidean space}} $\R^n = \R
\times \R \times \cdots \times \R$.   We use the following
notation for points: $x =(x_1,x_2,\ldots,x_n) \in \R^n$.  We also
simply write $0 \in \R^n$ to mean the vector $(0,0,\ldots,0)$.  Before
making $\R^n$ a metric space, let us prove an important inequality, the
so-called Cauchy-Schwarz inequality.

\begin{lemma}[\myindex{Cauchy-Schwarz inequality}]
Take $x =(x_1,x_2,\ldots,x_n) \in \R^n$ and $y =(y_1,y_2,\ldots,y_n) \in
\R^n$.  Then
\begin{equation*}
{\biggl( \sum_{j=1}^n x_j y_j \biggr)}^2
\leq
\biggl(\sum_{j=1}^n x_j^2 \biggr)
\biggl(\sum_{j=1}^n y_j^2 \biggr) .
\end{equation*}
\end{lemma}

\begin{proof}
Any square of a real number is nonnegative.  Hence any sum of squares is
nonnegative:
\begin{equation*}
\begin{split}
0 & \leq 
\sum_{j=1}^n \sum_{k=1}^n (x_j y_k - x_k y_j)^2
\\
& =
\sum_{j=1}^n \sum_{k=1}^n \bigl( x_j^2 y_k^2 + x_k^2 y_j^2 - 2 x_j x_k y_j
y_k \bigr)
\\
& =
\biggl( \sum_{j=1}^n x_j^2 \biggr)
\biggl( \sum_{k=1}^n y_k^2 \biggr)
+
\biggl( \sum_{j=1}^n y_j^2 \biggr)
\biggl( \sum_{k=1}^n x_k^2 \biggr)
-
2
\biggl( \sum_{j=1}^n x_j y_j \biggr)
\biggl( \sum_{k=1}^n x_k y_k \biggr)
\end{split}
\end{equation*}
We relabel and divide by 2 to obtain
\begin{equation*}
0 \leq 
\biggl( \sum_{j=1}^n x_j^2 \biggr)
\biggl( \sum_{j=1}^n y_j^2 \biggr)
-
{\biggl( \sum_{j=1}^n x_j y_j \biggr)}^2 ,
\end{equation*}
which is precisely what we wanted.
\end{proof}

\begin{example}
Let us construct
standard metric\index{standard metric on $\R^n$} for $\R^n$.  Define
\begin{equation*}
d(x,y) :=
\sqrt{
{(x_1-y_1)}^2 + 
{(x_2-y_2)}^2 + 
\cdots +
{(x_n-y_n)}^2
} =
\sqrt{
\sum_{j=1}^n
{(x_j-y_j)}^2 
} .
\end{equation*}
For $n=1$, the real line, this metric agrees with what we did above.  Again,
the only tricky part of the definition to check is the triangle inequality.
It is less messy to work with the square of the metric.  In the
following, note the use of the Cauchy-Schwarz inequality.
\begin{equation*}
\begin{split}
d(x,z)^2 & =
\sum_{j=1}^n
{(x_j-z_j)}^2 
\\
& =
\sum_{j=1}^n
{(x_j-y_j+y_j-z_j)}^2 
\\
& =
\sum_{j=1}^n
\Bigl(
{(x_j-y_j)}^2+{(y_j-z_j)}^2 + 2(x_j-y_j)(y_j-z_j)
\Bigr)
\\
& =
\sum_{j=1}^n
{(x_j-y_j)}^2
+
\sum_{j=1}^n
{(y_j-z_j)}^2 
+
\sum_{j=1}^n
 2(x_j-y_j)(y_j-z_j)
\\
& \leq
\sum_{j=1}^n
{(x_j-y_j)}^2
+
\sum_{j=1}^n
{(y_j-z_j)}^2 
+
2
\sqrt{
\sum_{j=1}^n
{(x_j-y_j)}^2
\sum_{j=1}^n
{(y_j-z_j)}^2
}
\\
& =
{\left(
\sqrt{
\sum_{j=1}^n
{(x_j-y_j)}^2
}
+
\sqrt{
\sum_{j=1}^n
{(y_j-z_j)}^2 
}
\right)}^2
=
{\bigl( d(x,y) + d(y,z) \bigr)}^2 .
\end{split}
\end{equation*}
Taking the square root of both sides we obtain the correct inequality.
\end{example}

\begin{example}
An example to keep in mind is the so-called \emph{\myindex{discrete
metric}}.
Let $X$ be any set and define
\begin{equation*}
d(x,y) :=
\begin{cases}
1 & \text{if $x \not= y$}, \\
0 & \text{if $x = y$}.
\end{cases}
\end{equation*}
That is, all points are equally distant from each other.  When $X$ is a
finite set, we can draw a diagram, see for example
\figureref{fig:msdiscmetric}.
Things become subtle when $X$ is an infinite set such
as the real numbers.
\begin{figure}[h!t]
\begin{center}
\input msdiscmetric.pdf_t
\caption{Sample discrete metric space $\{ a,b,c,d,e \}$, the distance
between any two points is $1$.\label{fig:msdiscmetric}}
\end{center}
\end{figure}

While this particular
example seldom comes up in practice, it is gives a useful 
``smell test.''  If you make a statement about metric spaces,
try it with the discrete metric.
To show that $(X,d)$ is indeed a metric space is left as an exercise.
\end{example}

\begin{example} \label{example:msC01}
Let $C([a,b])$ be the set of continuous real-valued functions on the
interval $[a,b]$.  Define the metric on $C([a,b])$ as
\begin{equation*}
d(f,g) := \sup_{x \in [a,b]} \abs{f(x)-g(x)} .
\end{equation*}
Let us check the properties.  First, $d(f,g)$ is finite as
$\abs{f(x)-g(x)}$ is a continuous function on a closed bounded interval
$[a,b]$, and so is bounded.
It is clear that $d(f,g) \geq 0$, 
it is the supremum of nonnegative numbers.  If $f = g$
then $\abs{f(x)-g(x)} = 0$ for all $x$ and hence $d(f,g) = 0$.  Conversely
if $d(f,g) = 0$, then for any $x$ we have $\abs{f(x)-g(x)} \leq d(f,g) = 0$
and hence $f(x) = g(x)$ for all $x$ and $f=g$.  That $d(f,g) = d(g,f)$
is equally trivial.  To show the triangle inequality we use the standard
triangle inequality.
\begin{equation*}
\begin{split}
d(f,h) & =
\sup_{x \in [a,b]} \abs{f(x)-g(x)} =
\sup_{x \in [a,b]} \abs{f(x)-h(x)+h(x)-g(x)}
\\
& \leq
\sup_{x \in [a,b]} ( \abs{f(x)-h(x)}+\abs{h(x)-g(x)} )
\\
& \leq
\sup_{x \in [a,b]} \abs{f(x)-h(x)}+
\sup_{x \in [a,b]} \abs{h(x)-g(x)} = d(f,h) + d(h,g) .
\end{split}
\end{equation*}
When treat $C([a,b])$ as a metric space without mentioning a metric, we mean this
particular metric.

This example may seem esoteric at first, but it turns out that working with
spaces such as $C([a,b])$ is really the meat of a large part of modern 
analysis.  Treating sets of functions as metric spaces allows us to
abstract away a lot of the grubby detail and prove powerful results such as
Picard's theorem with less work.
\end{example}

Oftentimes it is useful to consider a subset of a larger metric space
as a metric space.  We obtain the following proposition, which has
a trivial proof.

\begin{prop}
Let $(X,d)$ be a metric space and $Y \subset X$, then the restriction
$d|_{Y \times Y}$ is a metric on $Y$.
\end{prop}

\begin{defn}
If $(X,d)$ is a metric space, $Y \subset X$, and $d' := d|_{Y \times Y}$,
then $(Y,d')$ is said to be a \emph{\myindex{subspace}} of $(X,d)$.
\end{defn}

It is common to simply write $d$ for the metric on $Y$, as it is 
the restriction of the metric on $X$.  Sometimes we will say that $d'$ is
the \emph{\myindex{subspace metric}} and that $Y$ has the
\emph{\myindex{subspace topology}}.

\medskip

A subset of the real
numbers is bounded whenever all its elements are at most some fixed distance
from 0.
We can also define bounded sets in a metric space.
When dealing with an arbitrary metric space there may not be some
natural fixed point 0.  For the purposes of boundedness it does not matter.

\begin{defn}
Let $(X,d)$ be a metric space.  A subset $S \subset X$ is said to be
\emph{bounded}\index{bounded set} if there exists a $p \in X$ and a
$B \in \R$ such that
\begin{equation*}
d(p,x) \leq B \quad \text{for all $x \in S$}.
\end{equation*}
We say that $(X,d)$ is bounded if $X$ itself is a bounded subset.
\end{defn}

For example, the set of real numbers with the standard metric is not a
bounded metric space.  It is not hard to see that a
subset of the real numbers is bounded in the
sense of \chapterref{rn:chapter} if and only if it is bounded as a subset of the
metric space of real numbers with the standard metric.

On the other hand, if we take the real numbers with the discrete metric,
then we obtain a bounded metric space.  In fact, any set with the
discrete metric is bounded.

\subsection{Exercises}

\begin{exercise}
Show that for any set $X$, the discrete metric ($d(x,y) = 1$ if $x\not=y$ and
$d(x,x) = 0$) does give a metric space $(X,d)$.
\end{exercise}

\begin{exercise}
Let $X := \{ 0 \}$ be a set.  Can you make it into a metric space?
\end{exercise}

\begin{exercise}
Let $X := \{ a, b \}$ be a set.  Can you make it into two distinct metric
spaces?  (define two distinct metrics on it)
\end{exercise}

\begin{exercise}
Let the set $X := \{ A, B, C \}$ represent 3 buildings on campus.  Suppose we
wish to our distance to be the time it takes to walk from one building to
the other.
It takes 5 minutes either way between buildings $A$ and $B$.  However,
building $C$ is on a hill and it takes 10 minutes from $A$ and 15 minutes
from $B$ to get to $C$.  On the other hand it takes 5 minutes to go
from $C$ to $A$ and 7 minutes to go from $C$ to $B$, as we are going
downhill.  Do these distances define a metric?  If so, prove it, if not say
why not.
\end{exercise}

\begin{exercise}
Suppose that $(X,d)$ is a metric space and
$\varphi \colon [0,\infty] \to \R$ is an increasing function such that 
$\varphi(t) \geq 0$ for all $t$ and $\varphi(t) = 0$ if and only if
$t=0$.  Also suppose that $\varphi$ is \emph{\myindex{subadditive}},
that is $\varphi(s+t) \leq \varphi(s)+\varphi(t)$.
Show that with $d'(x,y) := \varphi\bigl(d(x,y)\bigr)$, we obtain a new
metric space $(X,d')$.
\end{exercise}

\begin{exercise}
Let $(X,d_X)$ and $(Y,d_Y)$ be metric spaces.\\
a) Show that $(X \times Y,d)$ with
$d\bigl( (x_1,y_1), (x_2,y_2) \bigr) := d_X(x_1,x_2) + d_Y(y_1,y_2)$ is
a metric space. \\
b) Show that $(X \times Y,d)$ with
$d\bigl( (x_1,y_1), (x_2,y_2) \bigr) := \max \{ d_X(x_1,x_2) , d_Y(y_1,y_2) \}$ is
a metric space.
\end{exercise}

\begin{exercise}
Let $X$ be the set of continuous functions on $[0,1]$.  Let $\varphi \colon
[0,1] \to (0,\infty)$ be continuous.  Define
\begin{equation*}
d(f,g) := \int_0^1 \abs{f(x)-g(x)}\varphi(x)~dx .
\end{equation*}
Show that $(X,d)$ is a metric space.
\end{exercise}

\pagebreak[2]

\begin{exercise}
Let $(X,d)$ be a metric space.  For nonempty bounded subsets $A$ and $B$ let
\begin{equation*}
d(x,B) := \inf \{ d(x,b) : b \in B \}
\qquad \text{and} \qquad
d(A,B) := \sup \{ d(a,B) : a \in A \} .
\end{equation*}
Now define the \emph{\myindex{Hausdorff metric}} as
\begin{equation*}
d_H(A,B) := \max \{ d(A,B) , d(B,A) \} .
\end{equation*}
Note: $d_H$ can be defined for arbitrary nonempty subsets if we allow the
extended reals.
\\
a) Let $Y \subset \sP(X)$ be the set of bounded nonempty subsets.  Show
that $(Y,d_H)$ is a metric space.
b) Show by example that $d$ itself is not a metric.  That is, $d$ is not
always symmetric.
\end{exercise}

%%%%%%%%%%%%%%%%%%%%%%%%%%%%%%%%%%%%%%%%%%%%%%%%%%%%%%%%%%%%%%%%%%%%%%%%%%%%%%

\sectionnewpage
\section{Open and closed sets}
\label{sec:mettop}

\sectionnotes{2 lectures}

\subsection{Topology}

It is useful to define a so-called \emph{\myindex{topology}}.  That is
we define closed and open sets in a metric space.  Before doing so,
let us define two special sets.

\begin{defn}
Let $(X,d)$ be a metric space, $x \in X$ and $\delta > 0$.  Then define
the \emph{\myindex{open ball}} or simply \emph{\myindex{ball}} of radius $\delta$
around $x$ as
\begin{equation*}
B(x,\delta) := \{ y \in X : d(x,y) < \delta \} .
\end{equation*}
Similarly we define the \emph{\myindex{closed ball}} as
\begin{equation*}
C(x,\delta) := \{ y \in X : d(x,y) \leq \delta \} .
\end{equation*}
\end{defn}

When we are dealing with different metric spaces, it is sometimes 
convenient to emphasize which metric space the ball is in.  We do this by
writing $B_X(x,\delta) := B(x,\delta)$ or $C_X(x,\delta) := C(x,\delta)$.

\begin{example}
Take the metric space $\R$ with the standard metric.  For
$x \in \R$, and $\delta > 0$ we get 
\begin{equation*}
B(x,\delta) = (x-\delta,x+\delta) \qquad \text{and} \qquad
C(x,\delta) = [x-\delta,x+\delta] .
\end{equation*}
\end{example}

\begin{example}
Be careful when working on a subspace.  Suppose we take the
metric space $[0,1]$ as a subspace of $\R$.  Then in $[0,1]$
we get
\begin{equation*}
B(0,\nicefrac{1}{2}) = B_{[0,1]}(0,\nicefrac{1}{2}) = [0,\nicefrac{1}{2}) .
\end{equation*}
This is of course different from $B_{\R}(0,\nicefrac{1}{2}) =
(-\nicefrac{1}{2},\nicefrac{1}{2})$.
The important thing to keep in mind is which metric space we are working
in.
\end{example}

\begin{defn}
Let $(X,d)$ be a metric space.  A set $V \subset X$
is \emph{open}\index{open set}
if for every $x \in V$, there exists a $\delta > 0$ such that
$B(x,\delta) \subset V$.  See \figureref{fig:msopenset}.  A set $E \subset X$ is 
\emph{closed}\index{closed set} if the complement $E^c = X \setminus E$ is open.
When the ambient space $X$ is not clear from context we say
$V$ is open in $X$ and $E$ is closed in $X$.

If $x \in V$ and $V$ is open, then we say that
$V$ is an \emph{\myindex{open neighborhood}} of $x$ (or
sometimes just \emph{\myindex{neighborhood}}).
\end{defn}

\begin{figure}[h!t]
\begin{center}
\input msopenset.pdf_t
\caption{Open set in a metric space.  Note that $\delta$ depends on $x$.\label{fig:msopenset}}
\end{center}
\end{figure}

Intuitively, an open set is a set that does not include its ``boundary.''
Note that not every set is either open or closed, in fact generally
most subsets are neither.

\begin{example}
The set $[0,1) \subset \R$ is neither open nor closed.  First,
every ball in $\R$ around $0$, $(-\delta,\delta)$ contains negative
numbers and hence is not contained in $[0,1)$ and so $[0,1)$ is not open.
Second, every ball in $\R$ around $1$, $(1-\delta,1+\delta)$ contains
numbers strictly less than 1 and greater than 0
(e.g.\ $1-\nicefrac{\delta}{2}$ as long as $\delta < 2$).
Thus $\R \setminus
[0,1)$ is not open, and so $[0,1)$ is not closed.
\end{example}

\begin{prop} \label{prop:topology:open}
Let $(X,d)$ be a metric space.
\begin{enumerate}[(i)]
\item \label{topology:openi} $\emptyset$ and $X$ are open in $X$.
\item \label{topology:openii} If $V_1, V_2, \ldots, V_k$ are open then
\begin{equation*}
\bigcap_{j=1}^k V_j
\end{equation*}
is also open.  That is, finite intersection of open sets is open.
\item \label{topology:openiii} If $\{ V_\lambda \}_{\lambda \in I}$ is
an arbitrary collection of open sets, then
\begin{equation*}
\bigcup_{\lambda \in I} V_\lambda
\end{equation*}
is also open.  That is, union of open sets is open.
\end{enumerate}
\end{prop}

Note that the index set in \eqref{topology:openiii} is arbitrarily large.
By $\bigcup_{\lambda \in I} V_\lambda$ we simply mean the set of
all $x$ such that $x \in V_\lambda$ for at least one $\lambda \in I$.

\begin{proof}
The set $X$ and $\emptyset$ are obviously open in $X$.

Let us prove \eqref{topology:openii}.
If $x \in \bigcap_{j=1}^k V_j$, then $x \in V_j$ for all $j$.
As $V_j$ are all open, there exists a $\delta_j > 0$ for every $j$
such that $B(x,\delta_j) \subset V_j$.  Take $\delta := \min \{
\delta_1,\ldots,\delta_k \}$ and note that $\delta > 0$.  We have
$B(x,\delta) \subset B(x,\delta_j) \subset V_j$ for every $j$ and thus
$B(x,\delta) \subset \bigcap_{j=1}^k V_j$.  Thus the intersection is open.

Let us prove \eqref{topology:openiii}.
If $x \in \bigcup_{\lambda \in I} V_\lambda$, then $x \in V_\lambda$ for some
$\lambda \in I$.
As $V_\lambda$ is open then there exists a $\delta > 0$
such that $B(x,\delta) \subset V_\lambda$.  But then
$B(x,\delta) \subset \bigcup_{\lambda \in I} V_\lambda$
and so the union is open.
\end{proof}

\begin{example}
The main thing to notice is the difference between
items
\eqref{topology:openii} and \eqref{topology:openiii}.
Item \eqref{topology:openii} is not true for an arbitrary intersection,
for example $\bigcap_{n=1}^\infty (-\nicefrac{1}{n},\nicefrac{1}{n}) = \{ 0
\}$, which is not open.
\end{example}


The proof of the following analogous proposition for closed sets
is left as an exercise.

\begin{prop} \label{prop:topology:closed}
Let $(X,d)$ be a metric space.
\begin{enumerate}[(i)]
\item \label{topology:closedi} $\emptyset$ and $X$ are closed in $X$.
\item \label{topology:closedii} If $\{ E_\lambda \}_{\lambda \in I}$ is
an arbitrary collection of closed sets, then
\begin{equation*}
\bigcap_{\lambda \in I} E_\lambda
\end{equation*}
is also closed.  That is, intersection of closed sets is closed.
\item \label{topology:closediii} If $E_1, E_2, \ldots, E_k$ are closed then
\begin{equation*}
\bigcup_{j=1}^k E_j
\end{equation*}
is also closed.  That is, finite union of closed sets is closed.
\end{enumerate}
\end{prop}

We have not yet shown that the open ball is open and the closed ball is
closed.  Let us show this fact now to justify the terminology.

\begin{prop} \label{prop:topology:ballsopenclosed}
Let $(X,d)$ be a metric space, $x \in X$, and $\delta > 0$.  Then
$B(x,\delta)$ is open and 
$C(x,\delta)$ is closed.
\end{prop}

\begin{proof}
Let $y \in B(x,\delta)$.  Let $\alpha := \delta-d(x,y)$.  Of course $\alpha
> 0$.  Now let $z \in B(y,\alpha)$.  Then
\begin{equation*}
d(x,z) \leq d(x,y) + d(y,z) < d(x,y) + \alpha = d(x,y) + \delta-d(x,y) =
\delta .
\end{equation*}
Therefore $z \in B(x,\delta)$ for every $z \in B(y,\alpha)$.  So $B(y,\alpha) \subset B(x,\delta)$ and
$B(x,\delta)$ is open.

The proof that $C(x,\delta)$ is closed is left as an exercise.
\end{proof}

Again be careful about what is the ambient metric space.
As $[0,\nicefrac{1}{2})$ is
an open ball in $[0,1]$, this means that $[0,\nicefrac{1}{2})$ is
an open set in $[0,1]$.  On the other hand $[0,\nicefrac{1}{2})$
is neither open nor closed in $\R$.

A useful way to think about an open set is a union of open balls.  If $U$ is
open, then for each $x \in U$, there is a $\delta_x > 0$ (depending on $x$
of course) such that
$B(x,\delta_x) \subset U$.  Then $U = \bigcup_{x\in U} B(x,\delta_x)$.

The proof of the following proposition is left as an exercise.  Note that there are other open and
closed sets in $\R$.

\begin{prop} \label{prop:topology:intervals:openclosed}
Let $a < b$ be two real numbers.  Then $(a,b)$, $(a,\infty)$,
and $(-\infty,b)$ are open in $\R$.
Also $[a,b]$, $[a,\infty)$,
and $(-\infty,b]$ are closed in $\R$.
\end{prop}


\subsection{Connected sets}

\begin{defn}
A nonempty
metric space $(X,d)$ is \emph{\myindex{connected}} if the
only subsets that are both open and closed are $\emptyset$ and $X$ itself.

When we apply the term \emph{connected} to a nonempty subset $A \subset X$, we simply
mean that $A$ with the subspace topology is connected.
\end{defn}

In other words, a nonempty $X$ is connected if whenever we write
$X = X_1 \cup X_2$ where $X_1 \cap X_2 = \emptyset$ and $X_1$ and $X_2$ are
open, then either $X_1 = \emptyset$ or $X_2 = \emptyset$.
So to test for disconnectedness, we need to find nonempty
disjoint open sets $X_1$ and
$X_2$ whose union is $X$.  For subsets, we state this idea as a proposition.

\begin{prop}
Let $(X,d)$ be a metric space.
A nonempty set $S \subset X$ is not connected if and only if
there exist open sets $U_1$ and
$U_2$ in $X$, such that $U_1 \cap U_2 \cap S = \emptyset$,
$U_1 \cap S \not= \emptyset$,
$U_2 \cap S \not= \emptyset$, and
\begin{equation*}
S = 
\bigl( U_1 \cap S \bigr)
\cup
\bigl( U_2 \cap S \bigr) .
\end{equation*}
\end{prop}

\begin{proof}
If $U_j$ is open in $X$,
then $U_j \cap S$ is open in $S$ in the subspace topology (with subspace
metric).  To see this,
note that if $B_X(x,\delta) \subset U_j$, then as
$B_S(x,\delta) = S \cap B_X(x,\delta)$,
we have $B_S(x,\delta) \subset U_j \cap S$.
The proof follows by the above discussion.

The proof of the other direction follows by using
\exerciseref{exercise:mssubspace} to find $U_1$ and $U_2$ from two
open disjoint subsets of $S$.
\end{proof}

\begin{example}
Let $S \subset \R$ be such that $x < z < y$ with $x,y \in S$
and $z \notin S$.  Claim: $S$ is not connected.  Proof:  Notice
\begin{equation}
\bigl( (-\infty,z) \cap S \bigr)
\cup
\bigl( (z,\infty) \cap S \bigr)
= S .
\end{equation}
\end{example}

\begin{prop}
A set $S \subset \R$ is connected if and only if it is
an interval or a single point.
\end{prop}

\begin{proof}
Suppose that $S$ is connected (so also nonempty).  If $S$ is a single point
then we are done.  So suppose that $x < y$ and $x,y \in S$.  If $z$ is such
that $x < z < y$, then $(-\infty,z) \cap S$ is nonempty and $(z,\infty) \cap
S$ is nonempty.  The two sets are disjoint.  As
$S$ is connected, we must have they their union is not $S$, so $z \in S$.

Suppose that $S$ is bounded, connected, but not a single point.
Let $\alpha := \inf S$ and
$\beta := \sup S$ and note that $\alpha < \beta$.  Suppose $\alpha < z < \beta$.  As $\alpha$ is the
infimum, then there is an $x \in S$ such that $\alpha \leq x < z$.  Similarly
there is a $y \in S$ such that $\beta \geq y > z$. 
We have shown above that $z \in S$, so $(\alpha,\beta) \subset S$.
If $w < \alpha$, then $w \notin S$
as $\alpha$ was the infimum,
similarly if $w > \beta$ then $w \notin S$.  Therefore the only
possibilities for $S$ are
$(\alpha,\beta)$,
$[\alpha,\beta)$,
$(\alpha,\beta]$,
$[\alpha,\beta]$.

The proof that an unbounded connected $S$ is an interval is left as an exercise.

On the other hand suppose that $S$ is an interval.
Suppose that $U_1$ and $U_2$ are open subsets of $\R$,
$U_1 \cap S$ and $U_2 \cap S$ are nonempty, and
$S = 
\bigl( U_1 \cap S \bigr)
\cup
\bigl( U_2 \cap S \bigr)$.  We will show that $U_1 \cap S$
and $U_2 \cap S$ contain a common point, so they are not disjoint,
and hence $S$ must be connected.
Suppose that there is $x \in U_1 \cap S$
and $y \in U_2 \cap S$.  We can assume that $x < y$.  As $S$ is an interval
$[x,y] \subset S$.  Let $z := \inf (U_2 \cap [x,y])$.
If $z = x$, then $z
\in U_1$.  If $z > x$,
then for any $\delta > 0$ the 
ball $B(z,\delta) =
(z-\delta,z+\delta)$ contains points that are not
in $U_2$, and so $z \notin U_2$ as $U_2$ is open.
Therefore, $z \in U_1$.
As $U_1$ is open, $B(z,\delta) \subset U_1$ for a small enough $\delta >
0$.
As $z$ is the infimum of $U_2 \cap [x,y]$, 
there must exist some $w \in U_2 \cap [x,y]$
such that $w \in [z,z+\delta) \subset B(z,\delta) \subset U_1$.
Therefore $w \in U_1 \cap U_2 \cap [x,y]$.
So $U_1 \cap S$ and $U_2 \cap S$ are not disjoint and hence $S$ is connected.
\end{proof}

\begin{example}
In many cases a ball $B(x,\delta)$ is connected.  But this is not
necessarily true in every metric space.
For a simplest example, take a two point space $\{ a,
b\}$ with the discrete metric.  Then $B(a,2) = \{ a , b \}$, which is not
connected as $B(a,1) = \{ a \}$ and 
$B(b,1) = \{ b \}$ are open and disjoint.
\end{example}

\subsection{Closure and boundary}

Sometime we wish to take a set and throw in everything that we can approach
from the set.  This concept is called the closure.

\begin{defn}
Let $(X,d)$ be a metric space and $A \subset X$.  Then
the \emph{\myindex{closure}} of $A$ is the set
\begin{equation*}
\overline{A} := \bigcap \{ E \subset X : \text{$E$ is closed and $A \subset
E$} \} .
\end{equation*}
That is, $\overline{A}$ is the intersection of all closed sets that contain
$A$.
\end{defn}

\begin{prop}
Let $(X,d)$ be a metric space and $A \subset X$.  The closure $\overline{A}$
is closed.  Furthermore if $A$ is closed then $\overline{A} = A$.
\end{prop}

\begin{proof}
First, the closure is the intersection of closed sets, so it is closed.
Second, if $A$ is closed, then take $E = A$, hence the intersection of all
closed sets $E$ containing $A$ must be equal to $A$.
\end{proof}

\begin{example}
The closure of $(0,1)$ in $\R$ is $[0,1]$.  Proof:  Simply notice that if
$E$ is closed and contains $(0,1)$, then $E$ must contain $0$ and $1$ (why?).
Thus $[0,1] \subset E$.  But $[0,1]$ is also closed.
Therefore the closure $\overline{(0,1)} = [0,1]$.
\end{example}

\begin{example}
Be careful to notice what ambient metric space you are working with.
If $X = (0,\infty)$, then
the closure of $(0,1)$ in $(0,\infty)$ is $(0,1]$.  Proof:  Similarly as
above $(0,1]$ is closed in $(0,\infty)$ (why?).  Any closed set $E$
that contains $(0,1)$ must contain 1 (why?).  Therefore $(0,1] \subset E$,
and hence $\overline{(0,1)} = (0,1]$ when working in $(0,\infty)$.
\end{example}

Let us justify the statement that the closure is everything that we can
``approach'' from the set.

\begin{prop} \label{prop:msclosureappr}
Let $(X,d)$ be a metric space and $A \subset X$.  Then $x \in \overline{A}$
if and only if for every $\delta > 0$, $B(x,\delta) \cap A \not=\emptyset$.
\end{prop}

\begin{proof}
Let us prove the two contrapositives.
Let us show that $x \notin \overline{A}$ if and only if there exists
a $\delta > 0$ such that $B(x,\delta) \cap A = \emptyset$.

First suppose that $x \notin \overline{A}$.  We know $\overline{A}$ is
closed.  Thus there is a $\delta > 0$ such that
$B(x,\delta) \subset \overline{A}^c$.  As $A \subset \overline{A}$ we
see that $B(x,\delta) \subset A^c$ and hence
$B(x,\delta) \cap A = \emptyset$.

On the other hand suppose that there is a $\delta > 0$ such that
$B(x,\delta) \cap A = \emptyset$.  Then $B(x,\delta)^c$ is a closed set and we have
that $A \subset B(x,\delta)^c$, but
$x \notin B(x,\delta)^c$.  Thus as $\overline{A}$ is the intersection
of closed sets containing $A$, we have $x \notin \overline{A}$.
\end{proof}

We can also talk about what is in the interior of a set and what is on the
boundary.

\begin{defn}
Let $(X,d)$ be a metric space and $A \subset X$, then
the \emph{\myindex{interior}} of $A$ is the set
\begin{equation*}
A^\circ := \{ x \in A : \text{there exists a $\delta > 0$ such that
$B(x,\delta) \subset A$} \} .
\end{equation*}
The \emph{\myindex{boundary}} of $A$ is the set
\begin{equation*}
\partial A := \overline{A}\setminus A^\circ.
\end{equation*}
\end{defn}

\begin{example}
Suppose $A=(0,1]$ and $X = \R$.  Then it is not hard
to see that $\overline{A}=[0,1]$, $A^\circ = (0,1)$,
and $\partial A = \{ 0, 1 \}$.
\end{example}

\begin{example}
Suppose $X = \{ a, b \}$ with the discrete metric.
Let $A = \{ a \}$, then $\overline{A} = A^\circ$ and $\partial A =
\emptyset$.
\end{example}


\begin{prop}
Let $(X,d)$ be a metric space and $A \subset X$.  Then $A^\circ$ is open
and $\partial A$ is closed.
\end{prop}

\begin{proof}
Given $x \in A^\circ$ we have $\delta > 0$ such that $B(x,\delta)
\subset A$.  If $z \in B(x,\delta)$, then as open balls are open,
there is an $\epsilon > 0$ such that $B(z,\epsilon) \subset B(x,\delta)
\subset A$, so $z$ is in $A^\circ$.  Therefore $B(x,\delta) \subset
A^\circ$ and so $A^\circ$ is open.

As $A^\circ$ is open, then
$\partial A = \overline{A} \setminus A^\circ = \overline{A} \cap
(A^\circ)^c$ is closed.
\end{proof}

The boundary is the set of points that are close to both the set and its
complement.

\begin{prop}
Let $(X,d)$ be a metric space and $A \subset X$.  Then $x \in \partial A$
if and only if for every $\delta > 0$,
$B(x,\delta) \cap A$ and
$B(x,\delta) \cap A^c$ are both nonempty.
\end{prop}

\begin{proof}
If $x \notin \overline{A}$, then there is some $\delta > 0$ such that
$B(x,\delta) \subset \overline{A}^c$ as $\overline{A}$ is closed.
So $B(x,\delta)$ contains no points of $A$.

Now suppose that $x \in A^\circ$, then there exists a $\delta > 0$
such that $B(x,\delta) \subset A$, but that means that $B(x,\delta)$
contains no points of $A^c$.

Finally suppose that $x \in \overline{A} \setminus A^\circ$.  Let $\delta >
0$ be arbitrary.  By \propref{prop:msclosureappr} $B(x,\delta)$ contains
a point from $A$.  Also, if $B(x,\delta)$ contained no points of $A^c$,
then $x$ would be in $A^\circ$.  Hence $B(x,\delta)$ contains a points of
$A^c$ as well.
\end{proof}

We obtain the following immediate corollary about closures of $A$ and $A^c$.  We
simply apply \propref{prop:msclosureappr}.

\begin{cor}
Let $(X,d)$ be a metric space and $A \subset X$.
Then $\partial A = \overline{A} \cap \overline{A^c}$.
\end{cor}

\subsection{Exercises}

\begin{exercise}
Prove \propref{prop:topology:closed}.  Hint: consider the complements of the
sets and apply \propref{prop:topology:open}.
\end{exercise}

\begin{exercise}
Finish the proof of \propref{prop:topology:ballsopenclosed} by
proving that $C(x,\delta)$ is closed.
\end{exercise}

\begin{exercise}
Prove \propref{prop:topology:intervals:openclosed}.
\end{exercise}

\begin{exercise}
Suppose that $(X,d)$ is a nonempty metric space with the discrete topology.  Show
that $X$ is connected if and only if it contains exactly one element.
\end{exercise}

\begin{exercise}
Show that if $S \subset \R$ is a connected unbounded set, then it is an
(unbounded) interval.
\end{exercise}

\begin{exercise}
Show that every open set can be written as a union of closed sets.
\end{exercise}

\begin{exercise}
a) Show that $E$ is closed if and only if $\partial E \subset E$.
b) Show that $U$ is open if and only if $\partial U \cap U = \emptyset$.
\end{exercise}

\begin{exercise}
a) Show that $A$ is open if and only if $A^\circ = A$.
b) Suppose that $U$ is an open set and $U \subset A$.  Show
that $U \subset A^\circ$.
\end{exercise}

\begin{exercise}
Let $X$ be a set and $d$, $d'$ be two metrics on $X$.
Suppose that there exists an $\alpha > 0$ and $\beta > 0$
such that $\alpha d(x,y) \leq d'(x,y) \leq \beta d(x,y)$ for all $x,y \in X$.
Show that $U$ is open in $(X,d)$ if and only if $U$ is open in $(X,d')$.
That is, the topologies of $(X,d)$ and $(X,d')$ are the same.
\end{exercise}


\begin{exercise}
Suppose that $\{ S_i \}$, $i \in \N$
is a collection of connected subsets of a metric space $(X,d)$.  Suppose
that there exists an $x \in X$ such that $x \in S_i$ for all $i \in N$.
Show that $\bigcup_{i=1}^\infty S_i$ is connected.
\end{exercise}

\begin{exercise}
Let $A$ be a connected set.
a) \nolinebreak Is $\overline{A}$ connected?  Prove or find a counterexample.
b) \nolinebreak Is $A^\circ$ connected?  Prove or find a counterexample.
Hint: Think of sets in $\R^2$.
\end{exercise}

\begin{exnote}
The definition of open sets in the following exercise is usually called the
\emph{\myindex{subspace topology}}.  You are asked to show that
we obtain the same topology by considering the subspace metric.
\end{exnote}

\begin{exercise} \label{exercise:mssubspace}
Suppose $(X,d)$ is a metric space and $Y \subset X$.  Show that
with the subspace metric on $Y$, a set $U \subset Y$
is open (in $Y$) whenever there exists an open set $V \subset X$ such
that $U = V \cap Y$.
\end{exercise}

\begin{exercise}
Let $(X,d)$ be a metric space.
a) For any $x \in X$ and $\delta > 0$, show
$\overline{B(x,\delta)} \subset C(x,\delta)$.
b) Is it always true that
$\overline{B(x,\delta)} = C(x,\delta)$?  Prove or find a counterexample.
\end{exercise}

\begin{exercise}
Let $(X,d)$ be a metric space and $A \subset X$.  Show that
$A^\circ = \bigcup \{ V : V \subset A \text{ is open} \}$.
\end{exercise}

%%%%%%%%%%%%%%%%%%%%%%%%%%%%%%%%%%%%%%%%%%%%%%%%%%%%%%%%%%%%%%%%%%%%%%%%%%%%%%

\sectionnewpage
\section{Sequences and convergence}
\label{sec:metseqs}

\sectionnotes{1 lecture}

\subsection{Sequences}

The notion of a sequence in a metric space is very similar to a sequence of
real numbers.

\begin{defn}
A \emph{\myindex{sequence}} in a metric space $(X,d)$ is a function
$x \colon \N \to X$.  As before we write $x_n$ for the $n$th element in
the sequence and use the notation $\{ x_n \}$, or more precisely
\begin{equation*}
\{ x_n \}_{n=1}^\infty .
\end{equation*}

A sequence $\{ x_n \}$ is \emph{bounded}\index{bounded sequence} if
there exists a point $p \in X$ and $B \in \R$ such that
\begin{equation*}
d(p,x_n) \leq B \qquad \text{for all $n \in \N$.}
\end{equation*}
In other words, the sequence $\{x_n\}$ is bounded whenever
the set $\{ x_n : n \in \N \}$
is bounded.

If $\{ n_j \}_{j=1}^\infty$ is a sequence of natural numbers
such that $n_{j+1} > n_j$ for all $j$ then
the sequence $\{ x_{n_j} \}_{j=1}^\infty$ is said to be
a \emph{\myindex{subsequence}} of $\{x_n \}$.
\end{defn}

Similarly we also define convergence.  Again, we will be cheating a little
bit and we will use the definite article in front of the word \emph{limit}
before we prove that the limit is unique.

\begin{defn}
A sequence $\{ x_n \}$ in a metric space $(X,d)$ is said
to \emph{\myindex{converge}} to a point
$p \in X$, if for every $\epsilon > 0$, there exists an $M \in \N$ such
that $d(x_n,p) < \epsilon$ for all $n \geq M$.  The point $p$
is said to be the \emph{limit}\index{limit of a sequence}
of $\{ x_n \}$.  We write
\begin{equation*}
\lim_{n\to \infty} x_n := p .
\end{equation*}

A sequence
that converges is said to be \emph{convergent}\index{convergent sequence}.
Otherwise, the sequence is said to be
\emph{divergent}\index{divergent sequence}.
\end{defn}

Let us prove that the limit is unique.  Note that the proof is almost
identical to the proof of the same fact for sequences of real numbers.
In fact many results we know for sequences of real numbers can be proved in
the more general settings of metric spaces.  We must replace $\abs{x-y}$
with $d(x,y)$ in the proofs and apply the triangle inequality correctly.

\begin{prop} \label{prop:mslimisunique}
A convergent sequence in a metric space has a unique limit.
\end{prop}

\begin{proof}
Suppose that the sequence $\{ x_n \}$ has the limit $x$ and the limit $y$.
Take an arbitrary $\epsilon > 0$.
From the definition we find an $M_1$ such that for all $n \geq M_1$,
$d(x_n,x) < \nicefrac{\epsilon}{2}$.  Similarly we find an $M_2$
such that for all $n \geq M_2$ we have
$d(x_n,y) < \nicefrac{\epsilon}{2}$.  Now take an $n$ such that
$n \geq M_1$ and also $n \geq M_2$
\begin{equation*}
\begin{split}
d(y,x)
& \leq
d(y,x_n) + d(x_n,x) \\
& <
\frac{\epsilon}{2} + \frac{\epsilon}{2} = \epsilon .
\end{split}
\end{equation*}
As $d(y,x) < \epsilon$ for all $\epsilon > 0$, then $d(x,y) = 0$
and $y=x$.  Hence the limit (if it exists) is unique.
\end{proof}

The proofs of the following propositions are left as exercises.

\begin{prop} \label{prop:msconvbound}
A convergent sequence in a metric space is bounded.
\end{prop}

\begin{prop} \label{prop:msconvifa}
A sequence $\{ x_n \}$ in a metric space $(X,d)$ converges to $p \in X$
if and only
if there exists a sequence $\{ a_n \}$ of real numbers such that
\begin{equation*}
d(x_n,p) \leq a_n \quad \text{for all $n \in \N$},
\end{equation*}
and
\begin{equation*}
\lim_{n\to\infty} a_n = 0.
\end{equation*}
\end{prop}

\subsection{Convergence in euclidean space}

It is useful to note what convergence means in the euclidean space
$\R^n$.

\begin{prop} \label{prop:msconveuc}
Let $\{ x^j \}_{j=1}^\infty$ be a sequence in $\R^n$,
where we write $x^j = \bigl(x_1^j,x_2^j,\ldots,x_n^j\bigr) \in \R^n$.
Then $\{ x^j \}_{j=1}^\infty$ converges if and only if
$\{ x_k^j \}_{j=1}^\infty$ converges for every $k$, in which case
\begin{equation*}
\lim_{j\to\infty}
x^j =
\Bigl(
\lim_{j\to\infty} x_1^j,
\lim_{j\to\infty} x_2^j,
\ldots,
\lim_{j\to\infty} x_n^j
\Bigr) .
\end{equation*}
\end{prop}

\begin{proof}
For $\R = \R^1$ the result is immediate.
So let $n > 1$.

Let $\{ x^j \}_{j=1}^\infty$ be a convergent sequence
in $\R^n$, where we write $x^j = \bigl(x_1^j,x_2^j,\ldots,x_n^j\bigr) \in \R^n$.
Let $x = (x_1,x_2,\ldots,x_n) \in \R^n$ be the limit.
Given $\epsilon > 0$, there exists an $M$ such that for all
$j \geq M$ we have
\begin{equation*}
d(x,x^j) < \epsilon.
\end{equation*}
Fix some $k=1,2,\ldots,n$.  For $j \geq M$ we have
\begin{equation}
\bigl\lvert x_k - x_k^j \bigr\rvert
=
\sqrt{{\bigl(x_k - x_k^j\bigr)}^2}
\leq
\sqrt{\sum_{\ell=1}^n {\bigl(x_\ell-x_\ell^j\bigr)}^2}
= d(x,x^j) < \epsilon .
\end{equation}
Hence the sequence $\{ x_k^j \}_{j=1}^\infty$ converges to $x_k$.

For the other direction suppose that
$\{ x_k^j \}_{j=1}^\infty$ converges to $x_k$ for every $k=1,2,\ldots,n$.
Hence, given $\epsilon > 0$, pick an $M$, such that if $j \geq M$ then 
$\bigl\lvert x_k-x_k^j \bigr\rvert < \nicefrac{\epsilon}{\sqrt{n}}$ for all
$k=1,2,\ldots,n$.  Then
\begin{equation*}
d(x,x^j)
=
\sqrt{\sum_{k=1}^n {\bigl(x_k-x_k^j\bigr)}^2}
<
\sqrt{\sum_{k=1}^n {\left(\frac{\epsilon}{\sqrt{n}}\right)}^2}
=
\sqrt{\sum_{k=1}^n \frac{{\epsilon^2}}{n}}
= \epsilon .
\end{equation*}
The sequence $\{ x^j \}$ converges to $x \in \R^n$ and we are done.
\end{proof}

\subsection{Convergence and topology}

The topology, that is, the set of open sets of a space encodes which
sequences converge.

\begin{prop} \label{prop:msconvtopo}
Let $(X,d)$ be a metric space and $\{x_n\}$ a sequence in $X$.  Then
$\{ x_n \}$ converges to $x \in X$ if and only if for every open neighborhood
$U$ of $x$, there exists an $M \in \N$ such that for all $n \geq M$
we have $x_n \in U$.
\end{prop}

\begin{proof}
First suppose that $\{ x_n \}$ converges.  Let $U$ be an open neighborhood
of $x$, then there exists an $\epsilon > 0$ such that $B(x,\epsilon) \subset
U$.  As the sequence converges, find an $M \in \N$ such that for all $n \geq
M$ we have $d(x,x_n) < \epsilon$ or in other words $x_n \in B(x,\epsilon)
\subset U$.

Let us prove the other direction.  Given $\epsilon > 0$ let $U :=
B(x,\epsilon)$ be the neighborhood of $x$.  Then there is an $M \in \N$
such that for $n \geq M$ we have $x_n \in U = B(x,\epsilon)$ or in other
words, $d(x,x_n) < \epsilon$.
\end{proof}

A set is closed when it contains the limits of its convergent sequences.

\begin{prop} \label{prop:msclosedlim}
Let $(X,d)$ be a metric space, $E \subset X$ a closed set
and $\{ x_n \}$ a sequence in $E$ that converges to some $x \in X$.
Then $x \in E$.
\end{prop}

\begin{proof}
Let us prove the contrapositive.
Suppose $\{ x_n \}$ is a sequence in $X$ that converges to $x \in E^c$.
As $E^c$ is open, \propref{prop:msconvtopo} says there is
an $M$ such that for all $n \geq M$,
$x_n \in E^c$.  So $\{ x_n \}$  is not a sequence in $E$.
%Let $\{ x_n \}$ be a sequence in $E$ that converges to some $x \in X$.
%Take $y \in E^c$, as $E^c$ is open then there exists a $\delta > 0$
%such that $B(y,\delta) \subset E^c$, or in particular $d(y,x_n) \geq \delta$
%for all $n$ as $x_n \in E$.  In particular, $\lim\, x_n = x \not= y$, and so
%$x \in E$.
\end{proof}

When we take a closure of a set $A$, we really throw in precisely 
those points that are limits of sequences in $A$.

\begin{prop} \label{prop:msclosureapprseq}
Let $(X,d)$ be a metric space and $A \subset X$.
If $x \in \overline{A}$, then there exists a sequence $\{ x_n \}$ of
elements in $A$ such that $\lim\, x_n = x$.
\end{prop}

\begin{proof}
Let $x \in \overline{A}$.  We know by
\propref{prop:msclosureappr} that given $\nicefrac{1}{n}$, there
exists a point $x_n \in B(x,\nicefrac{1}{n}) \cap A$.
As $d(x,x_n) < \nicefrac{1}{n}$, we have that $\lim\, x_n = x$.
\end{proof}

\subsection{Exercises}

\begin{exercise}
Let $(X,d)$ be a metric space and
let $A \subset X$.  Let $E$ be the set of all $x \in X$ such that there
exists a sequence $\{ x_n \}$ in $A$ that converges to $x$.  Show that
$E = \overline{A}$.
\end{exercise}

\begin{exercise}
a) Show that $d(x,y) := \min \{ 1, \abs{x-y} \}$ defines a metric on $\R$.
b) Show that a sequence converges in $(\R,d)$ if and only if it converges
in the standard metric.  c) Find a bounded sequence in $(\R,d)$ that
contains no convergent subsequence.
\end{exercise}

\begin{exercise}
Prove \propref{prop:msconvbound}
\end{exercise}

\begin{exercise}
Prove \propref{prop:msconvifa}
\end{exercise}

\begin{exercise}
Suppose that $\{x_n\}_{n=1}^\infty$ converges to $x$.  Suppose that $f \colon \N
\to \N$ is a one-to-one and onto function.  Show that
$\{ x_{f(n)} \}_{n=1}^\infty$ converges to $x$.
\end{exercise}

\begin{exercise}
If $(X,d)$ is a metric space where $d$ is the discrete metric.  Suppose that
$\{ x_n \}$ is a convergent sequence in $X$.  Show that there exists
a $K \in \N$ such that for all $n \geq K$ we have $x_n = x_K$.
\end{exercise}

\begin{exercise}
A set $S \subset X$ is said to be dense in $X$ if for every $x \in X$,
there exists a sequence $\{ x_n \}$ in $S$ that converges to $x$.  Prove
that $\R^n$ contains a countable dense subset.
\end{exercise}

\begin{exercise}[Tricky]
Suppose $\{ U_n \}_{n=1}^\infty$ be a decreasing ($U_{n+1} \subset U_n$ for
all $n$) sequence of open sets in a metric space $(X,d)$ such that
$\bigcap_{n=1}^\infty U_n = \{ p \}$ for some $p \in X$.  Suppose that
$\{ x_n \}$ is a sequence of points in $X$ such that $x_n \in U_n$.  Does
$\{ x_n \}$ necessarily converge to $p$?  Prove or construct a counterexample.
\end{exercise}

\begin{exercise}
Let $E \subset X$ be closed and
let $\{ x_n \}$ be a sequence in $X$ converging to $p \in X$.  Suppose
$x_n \in E$ for infinitely many $n \in \N$.  Show $p \in E$.
\end{exercise}

\begin{exercise}
Take $\R^* = \{ -\infty \} \cup \R \cup \{ \infty \}$ be the extended reals.
Define $d(x,y) := \frac{\abs{x-y}}{1+\abs{x-y}}$ if $x, y \in \R$,
define $d(\infty,x) := d(-\infty,x) = 1$ for all $x \in \R$, and
let $d(\infty,-\infty) := 2$.
a) Show that $(\R^*,d)$ is a metric space.
b) Suppose that $\{ x_n \}$ is a sequence of real numbers such that
for $x_n \geq n$ for all $n$.  Show that $\lim x_n = \infty$ in
$(\R^*,d)$.
\end{exercise}

\begin{exercise}
Suppose that $\{ V_n \}_{n=1}^\infty$ is a collection of open sets
in $(X,d)$
such that $V_{n+1} \supset V_n$.  Let $\{ x_n \}$ be a sequence
such that $x_n \in V_{n+1} \setminus V_n$ and suppose that
$\{ x_n \}$ converges to $p \in X$.  Show that $p \in \partial V$
where $V = \bigcup_{n=1}^\infty V_n$.
\end{exercise}

%%%%%%%%%%%%%%%%%%%%%%%%%%%%%%%%%%%%%%%%%%%%%%%%%%%%%%%%%%%%%%%%%%%%%%%%%%%%%%

\sectionnewpage
\section{Completeness and compactness}
\label{sec:metcompact}

\sectionnotes{2 lectures}

\subsection{Cauchy sequences and completeness}

Just like with sequences of real numbers we can define Cauchy sequences.

\begin{defn}
Let $(X,d)$ be a metric space.
A sequence $\{ x_n \}$ in $X$ is a \emph{\myindex{Cauchy sequence}} if
for every $\epsilon > 0$ there exists an $M \in \N$ such that
for all $n \geq M$ and all $k \geq M$ we have
\begin{equation*}
d(x_n, x_k) < \epsilon .
\end{equation*}
\end{defn}

The definition is again simply a translation of the concept
from the real numbers to metric spaces.  So a sequence of real
numbers is Cauchy in the sense of \chapterref{seq:chapter} if and only if
it is Cauchy in the sense above, provided we equip the real numbers with
the standard metric $d(x,y) = \abs{x-y}$.

\begin{defn}
Let $(X,d)$ be a metric space.  We say that $X$ is
\emph{\myindex{complete}} or \emph{\myindex{Cauchy-complete}}
if every Cauchy sequence $\{ x_n \}$ in $X$
converges to an $x \in X$.
\end{defn}

\begin{prop}
The space $\R^n$ with the standard metric is a complete metric space.
\end{prop}

\begin{proof}
For $\R = \R^1$ this was proved in \chapterref{seq:chapter}.

Take $n > 1$.  Let $\{ x^j \}_{j=1}^\infty$ be a Cauchy sequence
in $\R^n$, where we write $x^j = \bigl(x_1^j,x_2^j,\ldots,x_n^j\bigr) \in \R^n$.
As the sequence is Cauchy, given $\epsilon > 0$, there exists an $M$ such that for all
$i,j \geq M$ we have
\begin{equation*}
d(x^i,x^j) < \epsilon.
\end{equation*}

Fix some $k=1,2,\ldots,n$, for $i,j \geq M$ we have
\begin{equation}
\bigl\lvert x_k^i - x_k^j \bigr\rvert
=
\sqrt{{\bigl(x_k^i - x_k^j\bigr)}^2}
\leq
\sqrt{\sum_{\ell=1}^n {\bigl(x_\ell^i-x_\ell^j\bigr)}^2}
= d(x^i,x^j) < \epsilon .
\end{equation}
Hence the sequence $\{ x_k^j \}_{j=1}^\infty$ is Cauchy.  As $\R$ is
complete the sequence converges; there exists an $x_k \in \R$ such that
$x_k = \lim_{j\to\infty} x_k^j$.

Write $x = (x_1,x_2,\ldots,x_n) \in \R^n$.
By \propref{prop:msconveuc} we have that $\{ x^j \}$ converges
to $x \in \R^n$ and hence $\R^n$ is complete.
\end{proof}

\subsection{Compactness}

\begin{defn}
Let $(X,d)$ be a metric space and $K \subset X$. 
The set $K$ is set to be \emph{\myindex{compact}}
if for any collection
of open sets $\{ U_{\lambda} \}_{\lambda \in I}$ such that
\begin{equation*}
K \subset \bigcup_{\lambda \in I} U_\lambda ,
\end{equation*}
there exists a finite subset
$\{ \lambda_1, \lambda_2,\ldots,\lambda_k \} \subset I$
such that
\begin{equation*}
K \subset \bigcup_{j=1}^k U_{\lambda_j} .
\end{equation*}
\end{defn}

A collection of open sets $\{ U_{\lambda} \}_{\lambda \in I}$ as above is
said to be a \emph{\myindex{open cover}} of $K$.  So a way to say that
$K$ is compact is to say that \emph{every open cover of $K$ has a finite
\myindex{subcover}}.

\begin{prop}
Let $(X,d)$ be a metric space.  A compact set $K \subset X$ is closed and
bounded.
\end{prop}

\begin{proof}
First, we prove that a compact set is bounded.
Fix $p \in X$.  We have the open cover
\begin{equation*}
K \subset \bigcup_{n=1}^\infty B(p,n) = X .
\end{equation*}
If $K$ is compact, then there exists some set of indices
$n_1 < n_2 < \ldots < n_k$ such that
\begin{equation*}
K \subset \bigcup_{j=1}^k B(p,n_j) = B(p,n_k) .
\end{equation*}
As $K$ is contained in a ball, $K$ is bounded.

Next, we show a set that is not closed is not compact.  Suppose that
$\overline{K} \not= K$, that is, there is a point $x \in \overline{K}
\setminus K$.
If $y \not= x$, then for $n$
with $\nicefrac{1}{n} < d(x,y)$ we have
$y \notin C(x,\nicefrac{1}{n})$. Furthermore $x \notin K$, so
\begin{equation*}
K \subset \bigcup_{n=1}^\infty C(x,\nicefrac{1}{n})^c .
\end{equation*}
As a closed ball is closed, $C(x,\nicefrac{1}{n})^c$ is open, and
so we have an open cover.
If we take any
finite collection of indices $n_1 < n_2 < \ldots < n_k$, then 
\begin{equation*}
\bigcup_{j=1}^k C(x,\nicefrac{1}{n_j})^c 
=
C(x,\nicefrac{1}{n_k})^c 
\end{equation*}
As $x$ is in the closure,
we have $C(x,\nicefrac{1}{n_k}) \cap K \not= \emptyset$, so there is no
finite subcover and $K$ is not compact.
\end{proof}

We prove below that 
in finite dimensional euclidean space
every closed bounded set is compact.
So closed bounded sets
of $\R^n$ are examples of compact sets.
It is not true that in every metric space, closed and bounded is equivalent
to compact.  There are many metric spaces where closed and bounded is not
enough to give compactness, see for example
\exerciseref{exercise:msclbounnotcompt}.

A useful property of compact sets in a metric space is that every
sequence has a convergent subsequence.  Such sets are sometimes called
\emph{\myindex{sequentially compact}}.  Let us prove that in the
context of metric spaces, a set is compact if and only if it is sequentially
compact.

\begin{thm} \label{thm:mscompactisseqcpt}
Let $(X,d)$ be a metric space.  Then $K \subset X$ is a compact set if
and only if every sequence in $K$ has a subsequence converging to
a point in $K$.
\end{thm}

\begin{proof}
Let $K \subset X$ be a set and
$\{ x_n \}$ a sequence in $K$.  Suppose that for each $x \in K$,
there is a ball $B(x,\alpha_x)$ for some $\alpha_x > 0$ such that
$x_n \in B(x,\alpha_x)$ for only finitely many $n \in \N$.
Then
\begin{equation*}
K \subset \bigcup_{x \in K} B(x,\alpha_x) .
\end{equation*}
Any finite collection of these balls is going to contain only finitely many
$x_n$.  Thus for any finite collection of such balls there is an $x_n \in K$
that is not in the union.  Therefore, $K$ is not compact.

So if $K$ is compact,
then there exists an $x \in K$ such that
for any $\delta > 0$,
$B(x,\delta)$ contains $x_k$ for infinitely many $k \in \N$.
$B(x,1)$ contains some $x_k$ so let $n_1 := k$.
If $n_{j-1}$ is defined, then there must
exist a $k > n_{j-1}$ such that $x_k \in B(x,\nicefrac{1}{j})$, so define
$n_j := k$.  Notice that
$d(x,x_{n_j}) < \nicefrac{1}{j}$.  By \propref{prop:msconvifa},
$\lim\, x_{n_j} = x$.

For the other direction, suppose that every sequence in $K$
has a 
subsequence converging in $K$.
Take
an open cover $\{ U_\lambda \}_{\lambda \in I}$ of $K$.
For every $x \in K$, define
\begin{equation*}
\delta(x) := \sup \{ \delta \in (0,1) :
B(x,\delta) \subset U_\lambda \text{ for some } \lambda \in I \} .
\end{equation*}
As $\{ U_\lambda \}$ is an open cover of $K$,
$\delta(x) > 0$ for each $x \in K$.
By construction,
for any positive $\epsilon < \delta(x)$ there must exist a $\lambda \in I$
such that $B(x,\epsilon) \subset U_\lambda$.

Pick a $\lambda_0 \in I$ and look
at $U_{\lambda_0}$.  If $K \subset U_{\lambda_0}$, we stop as we have found a
finite subcover.
Otherwise, there must be
a point $x_1 \in K \setminus U_{\lambda_0}$.
There must exist some $\lambda_1 \in I$ such that $x_1 \in U_{\lambda_1}$
and in fact $B\bigl(x_1,\frac{1}{2}\delta({x_1})\bigr) \subset U_{\lambda_1}$.
We work inductively.  Suppose that $\lambda_{n-1}$ is defined.
Either
$U_{\lambda_0} \cup
U_{\lambda_1} \cup \cdots \cup
U_{\lambda_{n-1}}$ is a finite cover of $K$, in which case we
stop, or
there must be 
a point $x_n \in K \setminus \bigl( U_{\lambda_1} \cup
U_{\lambda_2} \cup \cdots \cup
U_{\lambda_{n-1}}\bigr)$.  In this case, there must be some $\lambda_n \in I$
such that $x_n \in U_{\lambda_n}$,
and in fact
\begin{equation*}
B\bigl(x_n,\tfrac{1}{2}\delta(x_n)\bigr) \subset U_{\lambda_n}.
\end{equation*}

So either we obtained a finite subcover or we obtained an
infinite
sequence $\{ x_n \}$ as above.  For contradiction suppose that
there was no finite subcover and we have the sequence $\{ x_n \}$.
Then there
is a subsequence $\{ x_{n_k} \}$ that converges, that
is, $x = \lim \, x_{n_k} \in K$.  We take $\lambda \in I$
such that $B\bigl(x,\frac{1}{2}\delta(x)\bigr) \subset U_\lambda$.  As the subsequence
converges, there is a $k$ such that $d(x_{n_k},x) < 
\frac{1}{8}\delta(x)$.
By the triangle inequality,
$B\bigl(x_{n_k},\frac{3}{8}\delta(x)\bigr) \subset
B\bigl(x,\frac{1}{2}\delta(x)\bigr) \subset U_\lambda$.
So $\frac{3}{8}\delta(x) < \delta({x_{n_k}})$, which implies
\begin{equation*}
B\bigl(x_{n_k},\tfrac{3}{16}\delta(x)\bigr)
\subset
B\bigl(x_{n_k},\tfrac{1}{2}\delta(x_{n_k})\bigr)
 \subset U_{\lambda_{n_k}}.
\end{equation*}
As $\nicefrac{1}{8} < \nicefrac{3}{16}$, we have
$x \in B\bigl(x_{n_k},\frac{3}{16}\delta(x)\bigr)$, or
$x \in U_{\lambda_{n_k}}$.  As
$\lim x_{n_j} = x$, for all $j$ large enough
we have $x_{n_j} \in U_{\lambda_{n_k}}$ by
\propref{prop:msconvtopo}.  Let us fix one of those $j$ such that $j > k$.
But by construction
$x_{n_j} \notin U_{\lambda_{n_k}}$ if $j > k$, which is a contradiction.
\end{proof}

\begin{example}
By the Bolzano-Weierstrass theorem for sequences (\thmref{thm:bwseq})
we have that any bounded sequence has a convergent
subsequence.  Therefore any sequence in a closed interval $[a,b] \subset \R$ has 
a convergent subsequence.  The limit must also be in $[a,b]$ as limits
preserve non-strict inequalities.  Hence a closed bounded interval $[a,b]
\subset \R$ is compact.
\end{example}

\begin{prop}
Let $(X,d)$ be a metric space and let $K \subset X$ be compact.  Suppose
that $E \subset K$ is a closed set, then $E$ is compact.
\end{prop}

\begin{proof}
Let $\{ x_n \}$ be a sequence in $E$.  It is also a sequence in $K$.
Therefore it has a convergent subsequence $\{ x_{n_j} \}$ that converges to
$x \in K$.  As $E$ is closed the limit of a sequence in $E$ is also in $E$
and so $x \in E$.  Thus $E$ must be compact.
\end{proof}

\begin{thm}[Heine-Borel%
\footnote{%
Named after the German mathematician 
Heinrich Eduard Heine (1821--1881),
and the French mathematician
F\'elix \'Edouard Justin \'Emile Borel (1871--1956).}]%
\index{Heine-Borel theorem}
\label{thm:msbw}
A closed bounded subset $K \subset \R^n$ is compact.
\end{thm}

\begin{proof}
For $\R = \R^1$ if $K \subset \R$ is closed and bounded, then
any sequence $\{ x_n \}$ in $K$ is bounded, so it has a convergent
subsequence by
Bolzano-Weierstrass theorem for sequences (\thmref{thm:bwseq}).
As $K$ is closed, the limit of the subsequence must be an element of
$K$.  So $K$ is compact.

Let us carry out the proof for $n=2$ and leave arbitrary $n$ as an exercise.

As $K$ is bounded, there exists a set
$B=[a,b]\times[c,d] \subset \R^2$ such that $K \subset B$.  If we can show
that $B$ is compact, then $K$, being a closed subset of a compact $B$, is
also compact.

Let $\{ (x_k,y_k) \}_{k=1}^\infty$ be a sequence in $B$.  That is,
$a \leq x_k \leq b$ and
$c \leq y_k \leq d$ for all $k$.  A bounded sequence has a convergent
subsequence so there is a subsequence $\{ x_{k_j} \}_{j=1}^\infty$
that is convergent.  The subsequence 
$\{ y_{k_j} \}_{j=1}^\infty$ is also a bounded sequence so there exists
a subsequence
$\{ y_{k_{j_i}} \}_{i=1}^\infty$ that is convergent.  A subsequence of a
convergent sequence is still convergent, so 
$\{ x_{k_{j_i}} \}_{i=1}^\infty$ is convergent.
Let
\begin{equation*}
x := \lim_{i\to\infty} x_{k_{j_i}}
\qquad \text{and} \qquad
y := \lim_{i\to\infty} y_{k_{j_i}} .
\end{equation*}
%Now write
%\begin{equation*}
%d\bigl((x_{k_{j_i}},y_{k_{j_i}}),(x,y)\bigr)
%=
%\sqrt{
%{\bigl(x_{k_{j_i}}-x\bigr)}^2 +
%{\bigl(y_{k_{j_i}}-y\bigr)}^2}.
%\end{equation*}
%As 
%$\sqrt{
%{\bigl(x_{k_{j_i}}-x\bigr)}^2 +
%{\bigl(y_{k_{j_i}}-y\bigr)}^2}$ goes to zero as $i$ goes to $\infty$,
%then
%$d\bigl((x_{k_{j_i}},y_{k_{j_i}}),(x,y)\bigr)$ goes to zero as
%$i$ goes to $\infty$.
By \propref{prop:msconveuc},
$\bigl\{ (x_{k_{j_i}},y_{k_{j_i}}) \bigr\}_{i=1}^\infty$ converges to $(x,y)$ as $i$ goes to $\infty$.
Furthermore, as $a \leq x_k \leq b$ and
$c \leq y_k \leq d$ for all $k$, we know that $(x,y) \in B$.
\end{proof}

\subsection{Exercises}

\begin{exercise}
Let $(X,d)$ be a metric space and $A$ a finite subset of $X$.
Show that $A$ is compact.
\end{exercise}

\begin{exercise}
Let $A = \{ \nicefrac{1}{n} : n \in \N \} \subset \R$.  a) Show that $A$ is
not compact directly using the definition.  b) Show that $A \cup \{ 0 \}$ is
compact directly using the definition.
\end{exercise}


\begin{exercise}
Let $(X,d)$ be a metric space with the discrete metric.  a) Prove that
$X$ is complete.  b) Prove that $X$ is compact if and only if $X$ is a finite
set.
\end{exercise}

\begin{exercise}
a) Show that the union of finitely many compact sets is a compact set.
b) Find an example where the union of infinitely many compact sets is not
compact.
\end{exercise}

\begin{exercise}
Prove \thmref{thm:msbw} for arbitrary dimension.
Hint: The trick is to use the correct notation.
\end{exercise}

\begin{exercise}
Show that a compact set $K$ is a complete metric space.
\end{exercise}

\begin{exercise}
Let $C([a,b])$ be the metric space as in \exampleref{example:msC01}.  Show that
$C([a,b])$ is a complete metric space.
\end{exercise}

\begin{exercise}[Challenging] \label{exercise:msclbounnotcompt}
Let $C([0,1])$ be the metric space of \exampleref{example:msC01}.  Let $0$
denote the zero function.  Then show that the closed ball
$C(0,1)$ is not compact (even though it is closed and bounded).
Hints: Construct a sequence of distinct continuous functions $\{ f_n \}$ such that
$d(f_n,0) = 1$ and $d(f_n,f_k) = 1$ for all $n \not= k$.  Show that
the set $\{ f_n : n \in \N \} \subset C(0,1)$ is closed but not compact.
See \chapterref{fs:chapter} for inspiration.
\end{exercise}

\begin{exercise}[Challenging]
Show that there exists a metric on $\R$ that makes $\R$ into a compact set.
\end{exercise}

\begin{exercise}
Suppose that $(X,d)$ is complete and suppose we have a countably infinite
collection of nonempty compact sets $E_1 \supset E_2 \supset E_3 \supset
\cdots$ then prove $\bigcap_{j=1}^\infty E_j \not= \emptyset$.
\end{exercise}

\begin{exercise}[Challenging]
Let $C([0,1])$ be the metric space of \exampleref{example:msC01}.
Let $K$ be the set of $f \in C([0,1])$ such that
$f$ is equal to a quadratic polynomial, i.e.\ $f(x) = a+bx+cx^2$, and such that
$\abs{f(x)} \leq 1$ for all $x \in [0,1]$,
that is $f \in C(0,1)$.  Show that $K$ is compact.
\end{exercise}

%%%%%%%%%%%%%%%%%%%%%%%%%%%%%%%%%%%%%%%%%%%%%%%%%%%%%%%%%%%%%%%%%%%%%%%%%%%%%%

\sectionnewpage
\section{Continuous functions}
\label{sec:metcont}

\sectionnotes{1 lecture}

\subsection{Continuity}

\begin{defn}
Let $(X,d_X)$ and $(Y,d_Y)$ be metric spaces and $c \in X$.
Then $f \colon X \to Y$ is
\emph{continuous at $c$}\index{continuous at $c$}
if for every $\epsilon > 0$
there is a $\delta > 0$ such that whenever $x \in X$ and $d_X(x,c) <
\delta$, then
$d_Y\bigl(f(x),f(c)\bigr) < \epsilon$.

\medskip

When $f \colon X \to Y$ is continuous at all $c \in X$, then we simply say
that $f$ is a \emph{\myindex{continuous function}}.
\end{defn}

The definition agrees with the definition from \chapterref{lim:chapter} when
$f$ is a real-valued function on the real line, when we take the standard
metric on $\R$.

\begin{prop} \label{prop:contiscont}
Let $(X,d_X)$ and $(Y,d_Y)$ be metric spaces and $c \in X$.
Then $f \colon X \to Y$ is
continuous at $c$
if and only if for every sequence $\{ x_n \}$ in $X$
converging to $c$, the sequence $\{ f(x_n) \}$ converges
to $f(c)$.
\end{prop}

\begin{proof}
Suppose that $f$ is continuous at $c$.  Let $\{ x_n \}$ be a
sequence in $X$ converging to $c$.  Given $\epsilon > 0$,
there is a $\delta > 0$ such that $d(x,c) < \delta$ implies
$d\bigl(f(x),f(c)\bigr) < \epsilon$.  So take $M$ such that
for all $n \geq M$, we have $d(x_n,c) < \delta$, then
$d\bigl(f(x_n),f(c)\bigr) < \epsilon$.  Hence $\{ f(x_n) \}$
converges to $f(c)$.

On the other hand suppose that $f$ is not continuous at $c$.
Then there exists an $\epsilon > 0$,
such that for every $\nicefrac{1}{n}$ there exists an $x_n \in X$,
$d(x_n,c) < \nicefrac{1}{n}$ such that $d\bigl(f(x_n),f(c)\bigr) \geq
\epsilon$.  Therefore $\{ f(x_n) \}$ does not converge to $f(c)$.
\end{proof}

\subsection{Compactness and continuity}

Continuous maps do not map closed sets to closed sets.  For example,
$f \colon (0,1) \to \R$ defined by $f(x) := x$ takes the set $(0,1)$, which
is closed in $(0,1)$, to the set $(0,1)$, which is not closed in $\R$.
On the other hand continuous maps do preserve compact sets.

\begin{lemma}
Let $(X,d_X)$ and $(Y,d_Y)$ be metric spaces,
and $f \colon X \to Y$ is a continuous function.  If
$K \subset X$ is a compact set, then $f(K)$ is a compact set.
\end{lemma}

\begin{proof}
Let $\{ f(x_n) \}_{n=1}^\infty$ be a sequence in $f(K)$,
then $\{ x_n \}_{n=1}^\infty$ is a sequence in $K$.  The set $K$ is compact and
therefore has a subsequence
$\{ x_{n_i} \}_{i=1}^\infty$ that converges to some $x \in K$.
By continuity,
\begin{equation*}
\lim_{i\to\infty} f(x_{n_i}) = f(x) \in f(K) .
\end{equation*}
Therefore every sequence in $f(K)$ has a subsequence convergent to 
a point in $f(K)$, so $f(K)$ is compact by \thmref{thm:mscompactisseqcpt}.
\end{proof}

As before, $f \colon X \to \R$ achieves an
\emph{\myindex{absolute minimum}} at $c \in X$ if
\begin{equation*}
f(x) \geq f(c) \qquad \text{ for all $x \in X$.}
\end{equation*}
On the other hand, $f$ achieves an 
\emph{\myindex{absolute maximum}} at $c \in X$ if
\begin{equation*}
f(x) \leq f(c) \qquad \text{ for all $x \in X$.}
\end{equation*}

\begin{thm}
Let $(X,d)$ and be a compact metric space,
and $f \colon X \to \R$ is a continuous function.  Then
$f(X)$ is compact and in fact
$f$ achieves an absolute minimum and an absolute maximum on $X$.
\end{thm}

\begin{proof}
As $X$ is compact and $f$ is continuous, we have
that $f(X) \subset \R$ is compact.  Hence $f(X)$ is closed
and bounded.  In particular,
$\sup f(X) \in f(X)$ and
$\inf f(X) \in f(X)$.  That is because both the sup and inf
can be achieved by sequences in $f(X)$ and $f(X)$ is closed.
Therefore there is some $x \in X$ such that $f(x) = \sup f(X)$
and some $y \in X$ such that $f(y) = \inf f(X)$.
\end{proof}

\subsection{Continuity and topology}

Let us see how to define continuity just in the terms of topology, that is,
the open sets.  We have already seen that topology determines which 
sequences converge, and so it is no wonder that the topology also
determines continuity of functions.

\begin{lemma} \label{lemma:mstopocontloc}
Let $(X,d_X)$ and $(Y,d_Y)$ be metric spaces.
A function $f \colon X \to Y$ is continuous at $c \in X$
if and only if for every open neighbourhood $U$ of $f(c)$ in $Y$, the set
$f^{-1}(U)$ contains an open neighbourhood of $c$ in $X$.
\end{lemma}

\begin{proof}
Suppose that $f$ is continuous at $c$.  
Let $U$ be an open neighbourhood of $f(c)$
in $Y$, then $B_Y\bigl(f(c),\epsilon\bigr) \subset U$ for some $\epsilon >
0$.  As $f$ is continuous, then there exists a $\delta > 0$
such that whenever $x$ is such that $d_X(x,c) < \delta$, then
$d_Y\bigl(f(x),f(c)\bigr) < \epsilon$.  In other words,
\begin{equation*}
B_X(c,\delta) \subset f^{-1}\bigl(B_Y\bigl(f(c),\epsilon\bigr)\bigr) .
\end{equation*}
and $B_X(c,\delta)$ is an open neighbourhood of $c$.

For the other direction,
let $\epsilon > 0$ be given.  If
$f^{-1}\bigl(B_Y\bigl(f(c),\epsilon\bigr)\bigr)$ contains an open
neighbourhood, it contains a ball, that is there is some $\delta > 0$
such that
\begin{equation*}
B_X(c,\delta) \subset f^{-1}\bigl(B_Y\bigl(f(c),\epsilon\bigr)\bigr) .
\end{equation*}
That means precisely that if $d_X(x,c) < \delta$ then $d_Y\bigl(f(x),f(c)\bigr)
< \epsilon$ and so $f$ is continuous at $c$.
\end{proof}

\begin{thm} \label{thm:mstopocont}
Let $(X,d_X)$ and $(Y,d_Y)$ be metric spaces.  A function $f \colon X \to Y$
is continuous if and only if
for every open $U \subset Y$, $f^{-1}(U)$ is open in $X$.
\end{thm}

The proof follows from \lemmaref{lemma:mstopocontloc} and is left as
an exercise.

\subsection{Exercises}

\begin{exercise}
Consider $\N \subset \R$ with the standard metric.  Let $(X,d)$ be a
metric space and $f \colon X \to \N$ a continuous function.  a) Prove that
if $X$ is connected, then $f$ is constant (the range of $f$ is a single
value).  b) Find an example where $X$ is disconnected and $f$ is not constant.
\end{exercise}

\begin{exercise}
Let $f \colon \R^2 \to \R$ be defined by $f(0,0) := 0$, and
$f(x,y) := \frac{xy}{x^2+y^2}$ if $(x,y) \not= (0,0)$.  a) Show that for any fixed $x$,
the function that takes $y$ to $f(x,y)$ is continuous.  Similarly
for any fixed $y$, the function that takes $x$ to $f(x,y)$ is continuous.
b) Show that $f$ is not continuous.
\end{exercise}

\begin{exercise} 
Suppose that $f \colon X \to Y$ is continuous for metric spaces $(X,d_X)$
and $(Y,d_Y)$.  Let $A \subset X$.  a) Show that $f(\overline{A}) \subset
\overline{f(A)}$.  b) Show that the subset can be proper.
\end{exercise}

\begin{exercise}
Prove \thmref{thm:mstopocont}.  Hint: Use \lemmaref{lemma:mstopocontloc}.
\end{exercise}

\begin{exercise} \label{exercise:msconnconn}
Suppose that $f \colon X \to Y$ is continuous for metric spaces $(X,d_X)$
and $(Y,d_Y)$.  Show that if $X$ is connected, then $f(X)$ is connected.
\end{exercise}

\begin{exercise}
Prove the following version of the
intermediate value theorem.  Let $(X,d)$ be a connected
metric space and $f \colon X \to \R$ a continuous function.  Suppose that
there exist $x_0,x_1 \in X$ and $y \in \R$ such that $f(x_0) < y < f(x_1)$.
Then prove that there exists a $z \in X$ such that $f(z) = y$.
Hint: see \exerciseref{exercise:msconnconn}.
\end{exercise}

\begin{exercise}
A continuous function $f \colon X \to Y$ for metric spaces $(X,d_X)$ and
$(Y,d_Y)$ is said to be \emph{\myindex{proper}}
if for every compact set $K \subset Y$, the set $f^{-1}(K)$ is compact.
Suppose that a continuous $f \colon (0,1) \to (0,1)$ is proper and $\{ x_n
\}$ is a sequence in $(0,1)$ that converges to $0$.  Show that
$\{ f(x_n) \}$ has no subsequence that converges in $(0,1)$.
\end{exercise}

\begin{exercise}
Let $(X,d_X)$ and $(Y,d_Y)$ be metric space and
$f \colon X \to Y$ be a one to one and onto continuous function.  Suppose
that $X$ is compact.  Prove that the inverse $f^{-1} \colon Y \to X$
is continuous.
\end{exercise}

\begin{exercise}
Take the metric space of continuous functions $C([0,1])$.  Let
$k \colon [0,1] \times [0,1] \to \R$ be a continuous function.
Given $f \in C([0,1])$ define
\begin{equation*}
\varphi_f(x) := \int_0^1 k(x,y) f(y) ~dy .
\end{equation*}
a) Show that $T(f) := \varphi_f$ defines a function $T \colon C([0,1]) \to
C([0,1])$.
b) Show that $T$ is continuous.
\end{exercise}

%%%%%%%%%%%%%%%%%%%%%%%%%%%%%%%%%%%%%%%%%%%%%%%%%%%%%%%%%%%%%%%%%%%%%%%%%%%%%%

\sectionnewpage
\section{Fixed point theorem and Picard's theorem again}
\label{sec:metpicard}

\sectionnotes{1--1.5 lecture (optional)}

In this section we prove a fixed point theorem for contraction
mappings.  As an application we prove Picard's theorem.  We have proved
Picard's theorem without metric spaces in \sectionref{sec:picard}.
The proof we present here is similar, but the proof goes a lot
smoother by using metric space concepts and the fixed point theorem.
For more examples on using Picard's theorem see \sectionref{sec:picard}.

\begin{defn}
Let $(X,d)$ and $(X',d')$ be metric spaces.
$F \colon X \to X'$ is said to be a \emph{contraction}
(or a contractive map) if it is
a $k$-Lipschitz map for some $k < 1$, i.e.\ if there exists a $k < 1$ such that
\begin{equation*}
d'\bigl(F(x),F(y)\bigr) \leq k d(x,y)
\ \ \ \ \text{for all } x,y \in X.
\end{equation*}

\medskip

If $T \colon X \to X$ is a map, $x \in X$ is called a \emph{fixed point}
if $T(x)=x$.
\end{defn}

\begin{thm}%
[\myindex{Contraction mapping principle} or \myindex{Fixed point theorem}]
\label{thm:contr}
Let $(X,d)$ be a nonempty complete metric space and $T \colon X \to X$ is a
contraction.
Then $T$ has a fixed point.
\end{thm}

Note that the words \emph{complete} and \emph{contraction} are necessary.
See \exerciseref{exercise:nofixedpoint}.

\begin{proof}
Pick any $x_0 \in X$.
Define a sequence $\{ x_n \}$ by $x_{n+1} := T(x_n)$.
\begin{equation*}
d(x_{n+1},x_n) = d\bigl(T(x_n),T(x_{n-1})\bigr)
\leq k d(x_n,x_{n-1})
\leq \cdots
\leq k^n d(x_1,x_0) .
\end{equation*}
So let $m \geq n$
\begin{equation*}
\begin{split}
d(x_m,x_n)
& \leq \sum_{i=n}^{m-1} d(x_{i+1},x_i) \\
& \leq \sum_{i=n}^{m-1} k^i d(x_1,x_0) \\
& = k^n d(x_1,x_0) \sum_{i=0}^{m-n-1} k^i \\
& \leq k^n d(x_1,x_0) \sum_{i=0}^{\infty} k^i
= k^n d(x_1,x_0) \frac{1}{1-k} .
\end{split}
\end{equation*}
In particular the sequence is Cauchy.  Since $X$ is complete
we let $x := \lim_{n\to \infty} x_n$ and claim that $x$
is our unique fixed point.

Fixed point?  Note that $T$ is continuous because it is a contraction.
Hence
\begin{equation*}
T(x) = \lim T(x_n) = \lim x_{n+1} = x .
\end{equation*}

Unique?  Let $y$ be a fixed point.
\begin{equation*}
d(x,y) = d\bigl(T(x),T(y)\bigr) = k d(x,y) .
\end{equation*}
As $k < 1$ this means that $d(x,y) = 0$ and hence $x=y$.  The theorem is
proved.
\end{proof}

Note that the proof is constructive.  Not only do we know that
a unique fixed point exists.  We also know how to find it.  Let us use the
theorem
to prove the classical Picard theorem on the existence and uniqueness of
ordinary differential equations.

Consider the equation
\begin{equation*}
\frac{dx}{dt} = F(t,x) .
\end{equation*}
Given some $t_0, x_0$ we are looking for a function $f(t)$ such that
$f'(t_0) = x_0$ and such that
\begin{equation*}
f'(t) = F\bigl(t,f(t)\bigr) .
\end{equation*}
There are some subtle issues.
Look at the equation $x' = x^2$, $x(0)=1$.  Then $x(t) = \frac{1}{1-t}$ is a
solution.  While $F$ is a reasonably ``nice'' function and in particular
exists for all $x$ and $t$, the solution ``blows up'' at $t=1$.

\begin{thm}[Picard's theorem on existence and uniqueness]%
\index{existence and uniqueness theorem}\index{Picard's theorem}
Let $I, J \subset \R$ be compact intervals and let $I_0$ and $J_0$
be their interiors.
Suppose $F \colon I \times J \to \R$ is continuous
and Lipschitz in the second variable, that is, there exists
$L \in \R$ such that
\begin{equation*}
\abs{F(t,x) - F(t,y)} \leq L \abs{x-y}
\ \ \ \text{ for all $x,y \in J$, $t \in I$} .
\end{equation*}
Let $(t_0,x_0) \in I_0 \times J_0$.
Then there exists $h > 0$ and a unique differentiable
$f \colon [t_0 - h, t_0 + h] \to \R$, such that
$f'(t) = F\bigl(t,f(t)\bigr)$ and $f(t_0) = x_0$.
\end{thm}

\begin{proof}
Without loss of generality assume $t_0 =0$.
Let $M := \sup \{ \abs{F(t,x)} : (t,x) \in I\times J \}$.  As $I \times J$ is
compact, $M < \infty$.  Pick $\alpha > 0$ such that
$[-\alpha,\alpha] \subset I$ and $[x_0-\alpha, x_0 + \alpha] \subset J$.
Let
\begin{equation*}
h := \min \left\{ \alpha, \frac{\alpha}{M+L\alpha} \right\} .
\end{equation*}
Note $[-h,h] \subset I$.  Define the set
\begin{equation*}
Y := \{ f \in C([-h,h]) : f([-h,h]) \subset [x_0-\alpha,x_0+\alpha] \} .
\end{equation*}
Here $C([-h,h])$ is equipped with the standard metric $d(f,g) := 
\sup \{ \abs{f(x)-g(x)} : x \in [-h,h] \}$.  With this metric
we have shown in an exercise that $C([-h,h])$ is a complete metric space.

\begin{exercise}
Show that $Y \subset \C([-h,h])$ is closed.
\end{exercise}

Define a mapping
$T \colon Y \to C([-h,h])$ by
\begin{equation*}
T(f)(t)
:=
x_0 + \int_0^t F\bigl(s,f(s)\bigr)~ds .
\end{equation*}

\begin{exercise}
Show that $T$ really maps into $C([-h,h])$.
\end{exercise}

Let $f \in Y$ and $\abs{t} \leq h$.
As $F$ is bounded by $M$ we have
\begin{equation*}
\begin{split}
\abs{T(f)(t) - x_0}
&= \abs{\int_0^t F\bigl(s,f(s)\bigr)~ds} \\
& \leq 
\abs{t}M \leq hM \leq \alpha .
\end{split}
\end{equation*}
Therefore, $T(Y) \subset Y$.  We can thus consider
$T$ as a mapping of $Y$ to $Y$.

We claim $T$ is a contraction.  First, for $t \in [-h,h]$
and $f,g \in Y$ we have
\begin{equation*}
\abs{F\bigl(t,f(t)\bigr) - F\bigl(t,g(t)\bigr)} \leq
L\abs{f(t)- g(t)} \leq L \, d(f,g) .
\end{equation*}
Therefore,
\begin{equation*}
\begin{split}
\abs{T(f)(t) - T(g)(t)}
&= \abs{\int_0^t F\bigl(s,f(s)\bigr) - F\bigl(s,g(s)\bigr)~ds} \\
& \leq \abs{t} L \, d(f,g) \\
& \leq h L\, d(f,g) \\
& \leq \frac{L\alpha}{M+L\alpha} \, d(f,g) .
\end{split}
\end{equation*}
We can assume $M > 0$ (why?).
Then $\frac{L\alpha}{M+L\alpha} < 1$ and the claim is proved.

Now apply the fixed point theorem (\thmref{thm:contr})
to find a unique $f \in Y$ such that $T(f) = f$, that is,
\begin{equation*}
f(t) = x_0 + \int_0^t F\bigl(s,f(s)\bigr)~ds .
\end{equation*}
By the fundamental theorem of calculus, $f$ is differentiable and
$f'(t) = F\bigl(t,f(t)\bigr)$.
\begin{exercise}
We have shown that $f$ is the unique function in $Y$.  Why is it the unique
continuous function $f \colon [-h,h] \to J$ that solves $T(f)=f$?  Hint:
Look at the last estimate in the proof.
\end{exercise}
\end{proof}

\subsection{Exercises}

\begin{exercise}
Suppose $X = X' = \R$ with the standard metric.  Let $0 < k < 1$, $b \in \R$.
a) Show that the map $F(x) = kx + b$ is a contraction.  b) Find the fixed
point and show directly that it is unique.
\end{exercise}

\begin{exercise}
Suppose $X = X' = [0,\nicefrac{1}{4}]$ with the standard metric.
a) Show that the map
$F(x) = x^2$ is a contraction, and find the best (largest) $k$ that works.
b) Find the fixed point and show directly that it is unique.
\end{exercise}

\begin{exercise} \label{exercise:nofixedpoint}
a) Find an example of a contraction of non-complete metric space with no fixed point.
b) Find a 1-Lipschitz map of a complete metric space with no fixed point.
\end{exercise}

\begin{exercise}
Consider $x' =x^2$, $x(0)=1$.  Start with $f_0(t) = 1$.  Find a 
few iterates (at least up to $f_2$).  Prove that
the limit of $f_n$ is $\frac{1}{1-t}$.
\end{exercise}


\end{document}
